% !TeX root = ../resumen.tex

\begin{document}
\section{Ordenes parciales}
\begin{definition}
  Una relacion binaria sobre $R$ sobre un conjunto $A$ sera llamada un \emph{orden parcial sobre} $A$,
  si es reflexiva, transitiva y antisimetrica respecto de $A$.
  
  Muchas veces denotaremos con $\leq$ a una relacion binaria que sea un orden parcial.

  Ademas, si hemos denotado $\leq$ a cierto orden parcial sobre un conjunto $A$, entonces:
  \begin{enumerate}
    \item Denotaremos con $<$ a la relacion binaria $\{(a, b) \in A^2 : a \leq b \text{ y } a \neq b\}$. Cuando se
    de que $a < b$, diremos que $a$ \emph{es menor que} $b$, o que $b$ \emph{es mayor que} $a$
    \item Denotaremos con $\prec$ a la relacion binaria $\{(a, b) \in A^2 : a < b \text{ y no existe $z$ tal que $a < z < b$}\}$.
    Cuando se de que $a \prec b$, diremos que $a$ \emph{es cubierto por} $b$ o que $b$ \emph{cubre a} $a$.
  \end{enumerate}
\end{definition}

\begin{definition}
  Un \emph{conjunto} parcialmente ordenado o poset, es un par \poset, donde $P$ es un conjunto no vacio
  cualquiera y $\leq$ es un orden parcial sobre $P$. Dado un poset \poset, el conjunto $P$ sera llamado el \emph{universo} de \poset.
\end{definition}

\subsection{Diagramas de Hasse}
Dado un poset \poset. con $P$ finito, podemos realizar un diagrama llamado \emph{diagrama de Hasse}, siguiendo las siguientes instrucciones:
\begin{enumerate}
  \item Asociar en forma inyectiva a cada $a \in P$ un punto $p_a$ del plano
  \item Trazar un segmento de recta uniendo los puntos $p_a$ y $p_b$, cada vez que $a \prec b$
  \item Realizar los antes dicho de tal forma que: \begin{enumerate}
    \item Si $a \prec b$, entonces $p_a$ esta por debajo de $p_b$
    \item Si un punto $p_a$ ocurre en un segmento del diagrama, entonces lo hace en alguno de sus extremos 
  \end{enumerate}
\end{enumerate}
La relacion de $\leq$ puede ser reconstruida facilmente apartir del diagrama. $a \leq b$ sucedera si y solo si
$p_a = p_b$ o hay una sucesion de caminos ascendentes de segmentos desde $p_a$ hasta $p_b$.

\subsection{Elementos maximales, maximos, minimales y minimos}
\begin{definition}
  Sea \poset un poset.
  
  Diremos que $a \in P$ es un elemento maximal de \poset, si no existe un $b \in P$ tal que $a < b$.

  Diremos que $a \in P$ es un elemento maximo de \poset si $b \leq a$, para todo $b \in P$.
  En caso de existir, sera denotado como 1, y muchas veces diremos que \poset tiene un 1 para expresar que \poset
  tiene un maximo

  Diremos que $a \in P$ es un elemento minimal de \poset, si no existe un $b \in P$ tal que $b < a$.
  
  Diremos que $a \in P$ es un elemento minimo de \poset si $a \leq b$ para todo $b \in P$.
  En caso de existir, sera denotado como 0, y muchas veces diremos que \poset tiene un 0 para expresar que \poset
  tiene un minimo
\end{definition}

\begin{remark}
  Un poset \poset tiene a lo sumo 1 maximo (resp. minimo)
\end{remark}
\begin{remark}
  Todo elemento maximo (resp. minimo) de \poset es un elemento maximal (resp. minimal) de \poset
\end{remark}

\subsection{Supremos}
Sea \poset un poset. Dado $S \subseteq P$, diremos que un elemento $a \in P$ es \emph{cota superior} de $S$
en \poset cuando $b \leq a$, para todo $b \in S$. Notese que todo elemento de $P$ es cota superior de $\emptyset$
en \poset. Un elemento $a \in P$ sera llamado \emph{supremo} de $S$ en \poset, cuando se den las siguientes propiedades:
\begin{enumerate}
  \item $a$ es cota superior de $S$ en \poset
  \item Para cada $b \in P$, si $b$ es cota superior de $S$ en \poset, entonces $a \leq b$
\end{enumerate}
\subsection{Infimos}
Sea \poset un poset. Dado $S \subseteq P$, diremos que un elemento $a \in P$ es \emph{cota inferior} de $S$
en \poset cuando $a \leq b$, para todo $b \in S$. Notese que todo elemento de $P$ es cota inferior de $\emptyset$
en \poset. Un elemento $a \in P$ sera llamado \emph{infimo} de $S$ en \poset, cuando se den las siguientes propiedades:
\begin{enumerate}
  \item $a$ es cota inferior de $S$ en \poset
  \item Para cada $b \in P$, si $b$ es cota inferior de $S$ en \poset, entonces $b \leq a$
\end{enumerate}
\begin{remark}
  Si $a$ es supremo (resp. infimo) de $S$ en \poset y $a'$ es supremo (resp. infimo) de $S$ en \poset, entonces $a = a'$
\end{remark}
\begin{remark}
  $a$ es supremo (resp. infimo) de $P$ en \poset $\iff a$ es maximo (resp. minimo) de \poset
\end{remark}

\subsection{Homomorfismos de posets}
\begin{definition}
Sea \poset y \posetdef{P'}{\leq'} posets. Una funcion $\functype{F}{P}{P'}$ sera llamada un \emph{homomorfismo de } \poset
en \posetdef{P'}{\leq'} si para todo $x, y \in P$ se cumple que $x \leq y$ implica $F(x) \leq' F(y)$.
Escribiremos \functype{F}{\poset}{\posetdef{P'}{\leq'}} para expresar que $F$ es un homomorfismo
de \poset en \posetdef{P'}{\leq'}
\end{definition}

\subsection{Isomorfismo de posets}
\begin{definition}
  Sea \poset y \posetdef{P'}{\leq'} posets. Una funcion \functype{F}{P}{P'} sera llamada un \emph{isomorfismo de}
  \poset en \posetdef{P'}{\leq'} si $F$ es biyectiva, F es un homomorfismo de \poset en \posetdef{P'}{\leq'}
  y $F^{-1}$ es un homomorfismo de \posetdef{P'}{\leq'} en \poset. Escribiremos \poset $\overset{\sim}{=}$ \posetdef{P'}{\leq'}
  cuando exista un isomorfismo de \poset en \posetdef{P'}{\leq'} y en tal caso diremos que \poset y \posetdef{P'}{\leq'}
  son isomorfos.
\end{definition}

\begin{definition}
  Dada una funcion \functype{F}{A}{B} y $S \subseteq A$, denotaremos con $F(S)$ al conjunto ${\{F(a) : a \in S\}}$
\end{definition}

\begin{lemma}
  Sean \poset y \posetdef{P'}{\leq'} posets. Supongamos $F$ es un isomorfismo de \poset y \posetdef{P'}{\leq'}
  \begin{enumerate}
    \item Para cada $S \subseteq P$ y cada $a \in P$, se tiene que $a$ es cota superior (resp. cota inferior) de $S \iff F(a)$
    es cota superior (resp. inferior) de $F(S)$
    \item Para cada $S \subseteq P$ y cada $a \in P$, se tiene que existe $\sup(S) \iff$ existe $\sup(F(S))$,
    y en el caso de que existan tales elementos, se tiene que $F(\sup(S)) = \sup(F(S))$
    \item Para cada $S \subseteq P$ y cada $a \in P$, se tiene que existe $\inf(S) \iff$ existe $\inf(F(S))$,
    y en el caso de que existan tales elementos, se tiene que $F(\inf(S)) = \inf(F(S))$
    \item Para cada $a \in P$, a es maximo (resp. minimo) $\iff F(a)$ es maximo (resp. minimo)
    \item Para cada $a \in P$, a es maximal (resp. minimal) $\iff F(a)$ es maximal (resp. minimal)
    \item Para $a, b \in P$, tenemos $a \prec b  \iff F(a) \prec' F(b)$
  \end{enumerate}
\end{lemma}
\begin{proof}
  (a) Supongamos $a$ es cota superior de S. Sea $s \in S$. Como $s \leq a$,
  tenemos que $F(s) \leq' F(a)$. Supongamos ahora que $F(a)$ es cota superior de $F(S)$. Sea $s \in S$.
  Como $F(s) \leq' F(a)$, tenemos que $s = F^{-1}(F(s)) \leq F^{-1}(F(a)) = a$.
  
  (b) Supongamos que existe $\sup(S)$. Entonces por (a) $F(\sup(S))$ es cota superior de $F(S)$.
  Supongamos $b$ es cota superior de $F(S)$, entonces $F^{-1}(b)$ es cota superior de $S$. Por lo tanto,
  $\sup(S) \leq F^{-1}(b)$ y entonces $F(\sup(S)) \leq' b$. La \emph{vuelta} es analoga.
  
  (c) La prueba es analoga a (b)
  
  (d) Supongamos $a \in P$ es maximo. Pero entonces $a = \sup(P)$, y entonces $F(a) = \sup(F(P)) = \sup(P')$.
  La \emph{vuelta} es analoga.

  (e) Supongamos $b \in P$ tal que no existe $a \in P$ tal que $b \leq a \Rightarrow a = b$.  Sea $c \in P$ tal que
  $F(b) \leq' F(c)$, entonces $b \leq c$, y entonces $b = c$, $F(b) = F(c)$. Luego $F(b)$ es maximal.
  La \emph{vuelta} es analoga.

  (f) Sean $a, b \in P$ tal que $a \prec b$. Luego tenemos que $F(a) \leq' F(b)$. Supongamos existe $z \in P$ tal que
  $F(a) \leq' F(z) \leq' F(b)$, entonces tendriamos $a \leq z \leq b$. Como $a \prec b$, se sigue que $z = a$ o $z = b$. Luego $F(a) \prec' F(b)$.
  La \emph{vuelta} es analoga.
\end{proof}

\subsection{Reticulados}
\begin{definition}
  Diremos que un poset \poset es un \emph{reticulado} si para todo $a, b \in P$, existen $\sup(\{a, b\})$ e $\inf(\{a, b\})$
\end{definition}

\begin{definition}
  Dado un reticulado \poset, definimos 2 operacion binarias:
  \funcdef{s}{P^2}{P}{(a,b)}{\sup(\{a, b\})}
  \funcdef{i}{P^2}{p}{(a,b)}{\inf(\{a, b\})}
Escribiremos \supbin{a}{b} en lugar de $s(a, b)$ y \infbin{a}{b} en lugar de $i(a, b)$
\end{definition}

\begin{lemma}
  Dado un reticulado \reticul y elementos $x, y \in L$, se cumplen:
  \begin{enumerate}
    \item $x \leq \supbin{x}{y}$
    \item $\infbin{x}{y} \leq x$
    \item $\supbin{x}{x} = \infbin{x}{x} = x$
    \item $\supbin{x}{y} = \supbin{y}{x}$
    \item $\infbin{x}{y} = \infbin{y}{x}$ 
  \end{enumerate}
\end{lemma}
\begin{proof}
  (1) Claramente (\supbin{x}{y}) es cota superior de $x$. Por lo tanto $x \leq \supbin{x}{y}$
  
  (2) Claramente (\infbin{x}{y}) es cota inferior de $x$. Por lo tanto $\infbin{x}{y} \leq x$
  
  (3) Supongamos $(\supbin{x}{x}) = z \neq x$. Tenemos que $z$ es cota superior de $x$ por lo tanto $x \leq z$.
  Pero tambien $x$ es cota superior de $x$ y por lo tanto $z$ no puede ser la minima cota superior. El caso del infimo es analogo.

  (4) $(\supbin{x}{y}) = \sup(\{x, y\}) = \sup(\{y, x\}) = (\supbin{y}{x})$

  (5) $(\infbin{x}{y}) = \inf(\{x, y\}) = \inf(\{y, x\}) = (\infbin{y}{x})$
\end{proof}

\begin{lemma}
  Dado un reticulado \reticul y elementos $x, y \in L$, son equivalentes:
  \begin{enumerate}
    \item $x \leq y$
    \item $\supbin{x}{y} = y$
    \item $\infbin{x}{y} = x$
  \end{enumerate}
\end{lemma}
\begin{proof}
  (1) $\Rightarrow$ (2) Claramente $y$ es cota superior de $\{x, y\}$ por (1). Y trivialmente es la minima ya que es igual a uno de sus elementos.

  (2) $\Rightarrow$ (3) Claremente $y$ es cota superior de $\{x, y\}$, por lo tanto $x \leq y$. Luego se tiene que $x$ es cota inferior de $\{x, y\}$. Trivialmente es la maxima.

  (3) $\Rightarrow$ (1) Claramente $x$ es cota inferior de $\{x, y\}$, por lo tanto $x \leq y$.
\end{proof}

\begin{lemma}[Leyes de absorcion]
  Dado un reticulado \reticul y elementos $x, y \in L$, se tiene que:
  \begin{enumerate}
    \item $\supbin{x}{(\infbin{x}{y})} = x$
    \item $\infbin{x}{(\supbin{x}{y})} = x$
  \end{enumerate}
\end{lemma}
\begin{proof}
  (1) Claramente $(\infbin{x}{y}) \leq x$, y por lo tanto $\supbin{(\infbin{x}{y})}{x} = x$

  (2) Claramente $x \leq (\supbin{x}{y})$, y por lo tanto $\infbin{x}{(\supbin{x}{y})} = x$
\end{proof}

\begin{lemma}
  Dado un reticulado \reticul y elementos $x, y, z \in L$, se tiene que:
  \begin{enumerate}
    \item \supbin{(\supbin{x}{y})}{z} = \supbin{x}{(\supbin{y}{z})}
    \item \infbin{(\infbin{x}{y})}{z} = \infbin{x}{(\infbin{y}{z})}
  \end{enumerate}
\end{lemma}
\begin{proof}
  (1) Notese que \supbin{x}{(\supbin{y}{z})} es cota superior de $\{x, y, z\}$ ya que:
  \begin{alignat*}{3}
    & x \leq \supbin{x}{(\supbin{y}{z})}\\
    & y \leq (\supbin{y}{z}) \leq \supbin{x}{(\supbin{y}{z})}\\
    & z \leq (\supbin{y}{z}) \leq \supbin{x}{(\supbin{y}{z})}
  \end{alignat*}
  En particular, tenemos que \supbin{x}{(\supbin{y}{z})} es cota superior de $\{x, y\}$, y entonces tenemos que
  $\supbin{x}{y} \leq \supbin{x}{(\supbin{y}{z})}$. Es decir, \supbin{x}{(\supbin{y}{z})} es cota superior de $\{\supbin{x}{y}, z\}$, y por lo tanto 
  $\supbin{(\supbin{x}{y})}{z} \leq \supbin{x}{(\supbin{y}{z})}$.

  Notese ahora que \supbin{(\supbin{x}{y})}{z} es cota superior de $\{x, y, z\}$, ya que:
  \begin{alignat*}{3}
    & x \leq (\supbin{x}{y}) \leq \supbin{(\supbin{x}{y})}{z}\\
    & y \leq (\supbin{x}{y}) \leq \supbin{(\supbin{x}{y})}{z}\\
    & z \leq \supbin{(\supbin{x}{y})}{z}\\
  \end{alignat*}
  En particular, tenemos que \supbin{(\supbin{x}{y})}{z} es cota superior de $\{y, z\}$, y por lo tanto $\supbin{y}{z} \leq \supbin{(\supbin{x}{y})}{z}$.
  Es decir, \supbin{(\supbin{x}{y})}{z} es cota superior de $\{x, \supbin{y}{z}\}$, y por lo tanto $\supbin{x}{(\supbin{y}{z})} \leq \supbin{(\supbin{x}{y})}{z}$.
  
  Luego tenemos que \supbin{(\supbin{x}{y})}{z} = \supbin{x}{(\supbin{y}{z})}

  (2) Es analoga, si alguien la quiere hacer
\end{proof}

\begin{lemma}
  Dado un reticulado \reticul y elementos $x, y, z, w \in L$, se tiene que
  si $x \leq z$ y $y \leq w$, entonces
  \begin{enumerate}
    \item $\supbin{x}{y} \leq \supbin{z}{w}$
    \item $\infbin{x}{y} \leq \infbin{z}{w}$
  \end{enumerate}
\end{lemma}
\begin{proof}
  (1) Notese que
  \begin{alignat*}{3}
    &x \leq z \leq \supbin{z}{w}\\
    &y \leq w \leq \supbin{z}{w}
  \end{alignat*}
  Luego \supbin{z}{w} es cota superior de $\{x, y\}$ y por lo tanto $\supbin{x}{y} \leq \supbin{z}{w}$

  (2) Notese que
  \begin{alignat*}{3}
    &z \geq x \geq \infbin{x}{y}\\
    &w \geq y \geq \infbin{x}{y}
  \end{alignat*}
  Luego $\infbin{x}{y}$ es cota inferior de $\{z, w\}$ y por lo tanto $\infbin{x}{y} \leq \infbin{z}{w}$
\end{proof}
\begin{lemma}
  Dado un reticulado \reticul y elementos $x, y, z \in L$, se tiene que\\
  $\supbin{(\infbin{x}{y})}{(\infbin{x}{z})} \leq \infbin{x}{(\supbin{y}{z})}$
\end{lemma}
\begin{proof}
  Notese que
  \begin{alignat*}{3}
    &(\infbin{x}{y}), (\infbin{x}{z}) \leq x\\
    &(\infbin{x}{y}), (\infbin{x}{z}) \leq \supbin{y}{z}
  \end{alignat*}
  Tenemos entonces que $(\infbin{x}{y}), (\infbin{x}{z}) \leq \infbin{x}{(\supbin{y}{z})}$ y por lo tanto $\supbin{(\infbin{x}{y})}{(\infbin{x}{z})} \leq \infbin{x}{(\supbin{y}{z})}$
\end{proof}
\begin{lemma}
  Sea \reticul un reticulado. Dados elementos $\succession{x}{1}{n} \in L$, con $n \geq 2$,
  se tiene que:
  \begin{alignat*}{2}
    &\supbin{(\dots\supbin{(\supbin{x_1}{x_2})}{\dots})}{x_n} &= \sup({\{\succession{x}{1}{n}}\})\\
    &\infbin{(\dots\infbin{(\infbin{x_1}{x_2})}{\dots})}{x_n} &= \inf({\{\succession{x}{1}{n}}\})        
  \end{alignat*}
\end{lemma}
\begin{proof}
  TODO
\end{proof}
\end{document}