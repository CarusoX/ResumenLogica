% !TeX root = ../resumen.tex

\section{Teorema de la completitud}

\begin{lemma}[Tipos parecidos]
  Sean $\tau=(\mathcal{C},\mathcal{F},\mathcal{R},a)$ y $\tau'=(\mathcal{C}',\mathcal{F}',\mathcal{R}',a')$ tipos. Se cumplen: \begin{enumerate}
    \item Si $\mathcal{C}\subseteq\mathcal{C}', \mathcal{F}\subseteq\mathcal{F}',\mathcal{R}\subseteq\mathcal{R}'$ y $a'|_{\mathcal{F}\cup\mathcal{R}}=a$,
    entonces $\forder\proves\varphi$ implica $(\Sigma,\tau')\proves\varphi$
    \item Si $\mathcal{C}\subseteq\mathcal{C}', \mathcal{F}\subseteq\mathcal{F}',\mathcal{R}\subseteq\mathcal{R}'$ y $a'=a$, entonces $(\Sigma,\tau')\proves\varphi$
    implica $\forder\proves\varphi$, cada vez que $\Sigma\cup\{\varphi\}\subseteq S^\tau$
  \end{enumerate} 
\end{lemma}
\noproof

\begin{lemma}[Lema del infimo]
  Sea $T=\forder$ una teoria y supongamos que $\tau$ tiene una cantidad infinita de nombres de constante que no ocurren en las sentencias de $Sigma$.
  Entonces para cada formula $\varphi=_d\varphi(v)$, se tiene que en el algebra de Lindembaym $\mathcal{A}_T$:
  $$
  [\forall v\varphi(v)]_T=\inf(\{[\varphi]_T:t\in T_c^\tau\})
  $$
  
\end{lemma}

\noproof

\begin{lemma}[Lema de Coincidencia]
  Sea $\tau=(\mathcal{C},\mathcal{F},\mathcal{R},a)$ y $\tau'=(\mathcal{C}',\mathcal{F}',\mathcal{R}',a')$ dos tipos cualesquiera y sea $\tau_\cap=(\mathcal{C}_\cap,\mathcal{F}_\cap,\mathcal{R}_\cap,a_\cap)$ donde:
  \begin{alignat*}{3}
    &\mathcal{C}_\cap&\ =&\ \mathcal{C}\cap\mathcal{C}'\\
    &\mathcal{F}_\cap&\ =&\ \{f\in\mathcal{F}\cap\mathcal{F}':a(f)=a'(f)\}\\
    &\mathcal{R}_\cap&\ =&\ \{r\in\mathcal{R}\cap\mathcal{R}':a(r)=a'(r)\}\\
    &a_\cap&\ =&\ a|_{\mathcal{F}_\cap\cup\mathcal{R}_\cap}
  \end{alignat*}
  Entonces $\tau_\cap$ es un tipo tal que $T^{\tau_\cap}=T^\tau\cap T^{\tau'}$ y $F^{\tau_\cap}=F^\tau\cap F^{\tau'}$. Sean $\mathbf{A}$
  y $\mathbf{A'}$ modelos de tipo $\tau$ y $\tau'$ respectivamente. Supongamos $A=A'$ y que $c^\mathbf{A}=c^\mathbf{A'}$, para cada $c\in\mathcal{C}_\cap$,
  $f^\mathbf{A}=f^\mathbf{A'}$, para cada $f\in\mathcal{F}_\cap$ y $r^\mathbf{A}=r^\mathbf{A'}$, para cada $r\in\mathcal{R}_\cap$. Entonces se cumplen: \begin{enumerate}
    \item Para cada $t=_d t(\vec{v})\in T^{\tau_\cap}$ se tiene que $t^\mathbf{A}[\vec{a}]=t^\mathbf{A'}[\vec{a}]$, para cada $\vec{a}\in A^n$
    \item Para cada $\varphi =_d \varphi(\vec{v})\in F^{\tau_\cap}$ se tiene que $\mathbf{A}\models\varphi[\vec{a}]\iff\mathbf{A'}\models\varphi[\vec{a}]$
    \item Si $\Sigma\cup\{\varphi\}\subseteq S^{\tau_\cap}$, entonces $\forder\models\varphi\iff(\Sigma,\tau')\models\varphi$
  \end{enumerate}
\end{lemma}

\noproof

\begin{lemma}
  Sea $\tau$ un tipo. Hay una infinitupla $(\gamma_1,\gamma_2,\dots)\in {F^\tau}^\mathbf{N}$ tal que:\begin{enumerate}
    \item $|Li(\gamma_j)|\leq 1$, para cada $j=1,2,\dots$
    \item Si $|Li(\gamma)|\leq 1$, entonces $\gamma=\gamma_j$ para algun $j\in\mathbf{N}$
  \end{enumerate}
\end{lemma}
\begin{proof}
  Notese que las formulas de tipo $\tau $ son palabras de algun alfabeto
finito $A$. Dado un orden total $\leq $ para $A$, podemos definir%
\begin{eqnarray*}
\gamma _{1} &=&\min\nolimits_{\alpha }^{\leq }\left( \alpha \in F^{\tau
}\wedge \left\vert Li(\alpha )\right\vert \leq 1\right)  \\
\gamma _{t+1} &=&\min\nolimits_{\alpha }^{\leq }\left( \alpha \in F^{\tau
}\wedge \left\vert Li(\alpha )\right\vert \leq 1\wedge (\forall i\in \mathbf{N}
)_{i\leq t}\alpha \neq \gamma _{i}\right) 
\end{eqnarray*}%
Notese que para $t\in \mathbf{N}$, tenemos que $\gamma _{t}=t$-esimo
elemento de $\{\alpha \in F^{\tau }:\left\vert Li(\alpha )\right\vert \leq
1\}$, con respecto al orden total de $A^{\ast }$ inducido por $\leq $.
Claramente entonces la infinitupla $(\gamma _{1},\gamma _{2},...)$ cumple
(1) y (2).
\end{proof}

\begin{theorem}[Teorema de Completitud]
  Sea $T=\forder$ una teoria de primer orden. Si $T\models\varphi$, entonces $T\proves\varphi$
\end{theorem}
\begin{proof}
  Primero probaremos completitud para el caso en que $\tau $ tiene una
cantidad infinita de nombres de constante que no ocurren en las sentencias de $
\Sigma$. Lo probaremos por el absurdo, es decir supongamos que hay una
sentencia $\varphi_0$ tal que $T\models\varphi_0$ y $T\not\vdash
\varphi_0$. Notese que ya que $T\not\vdash\varphi_0$, tenemos que $
[\neg\varphi_0]_{T}\not=0^{T}$ (sino, $\neg\varphi_0$ seria refutable, y por lo tanto se podria $T\proves\varphi_0$). Por el lema anterior hay una
infinitupla $(\gamma _{1},\gamma _{2},...)\in F^{\tau \mathbf{N}}$ tal que:

\begin{enumerate}
\item[-] $\left\vert Li(\gamma _{j})\right\vert \leq 1$, para cada $
j=1,2,... $

\item[-] Si $\left\vert Li(\gamma )\right\vert \leq 1$, entonces $\gamma
=\gamma _{j}$, para algun $j\in \mathbf{N}$
\end{enumerate}

\noindent Para cada $j\in \mathbf{N}$, sea $w_{j}\in Var$ tal que $Li(\gamma
_{j})\subseteq \{w_{j}\}$. Para cada $j$, declaremos $\gamma _{j}=_{d}\gamma
_{j}(w_{j})$. Notese que por el Lema del infimo tenemos que $\inf
\{[\gamma _{j}(t)]_{T}:t\in T_{c}^{\tau }\}=[\forall w_{j}\gamma
_{j}(w_{j})]_{T}$, para cada $j=1,2,...$. Por el Teorema de Rasiova y
Sikorski tenemos que hay un filtro primo $\mathcal{U}$ de $\mathcal{A}_{T}$,
el cual cumple:

\begin{enumerate}
\item[(a)] $[\neg \varphi_0]_{T}\in \mathcal{U}$

\item[(b)] Para cada $j\in \mathbf{N}$, $\{[\gamma _{j}(t)]_{T}:t\in
T_{c}^{\tau }\}\subseteq \mathcal{U}$ implica que $[\forall w_{j}\gamma
_{j}(w_{j})]_{T}\in \mathcal{U}$
\end{enumerate}

\noindent Ya que la infinitupla $(\gamma _{1},\gamma _{2},...)$ cubre todas
las formulas con a lo sumo una variable libre, podemos reescribir la
propiedad (b) de la siguiente manera

\begin{enumerate}
\item[(b)$^{\prime}$] Para cada $\varphi =_{d}\varphi (v)\in F^{\tau }$, si 
$\{[\varphi(t)]_{T}:t\in T_{c}^{\tau }\}\subseteq \mathcal{U}$ entonces $%
[\forall v\varphi (v)]_{T}\in \mathcal{U}$
\end{enumerate}

\noindent Definamos sobre $T_{c}^{\tau }$ la siguiente relacion:%
\begin{equation*}
t\bowtie s\text{ si y solo si }[(t\equiv s)]_{T}\in \mathcal{U}\text{.}
\end{equation*}%
Veamos entonces que:

\begin{enumerate}
\item $\bowtie$ es de equivalencia.

\item Para cada $\varphi =_{d}\varphi (v_{1},...,v_{n})\in F^{\tau }$, $
t_{1},...,t_{n},s_{1},...,s_{n}\in T_{c}^{\tau }$, si $t_{1}\bowtie s_{1}$, $
t_{2}\bowtie s_{2}$, $...$, $t_{n}\bowtie s_{n}$, entonces $[\varphi
(t_{1},...,t_{n})]_{T}\in \mathcal{U}$ si y solo si $[\varphi
(s_{1},...,s_{n})]_{T}\in \mathcal{U}$.

\item Para cada $f\in \mathcal{F}_{n}$, $
t_{1},...,t_{n},s_{1},...,s_{n}\in T_{c}^{\tau }$,
$$
t_{1}\bowtie s_{1},t_{2}\bowtie s_{2},...,\;t_{n}\bowtie s_{n}\text{ implica 
}f(t_{1},...,t_{n})\bowtie f(s_{1},...,s_{n}).
$$
\end{enumerate}

Probaremos (1), mostrando que valen las 3 propiedades.
\begin{itemize}
  \item $t \bowtie t \iff [t\equiv t]_T \in \mathcal{U} \iff 1^T \in \mathcal{U}$, trivial pues $\mathcal{U}$ es filtro.
  \item $(t \bowtie s \implies s \bowtie t) \iff ([t\equiv s]_T \in \mathcal{U} \implies [s\equiv t]_T \in \mathcal{U})$, es trivial que $[t\equiv s]_T = [s\equiv t]_T$ (se deduce por commutatividad).
  \item $(t\bowtie s, s \bowtie r \implies t\bowtie r) \iff ([t\equiv s]_T \in \mathcal{U}, [s \equiv r]_T \in \mathcal{U} \implies [t \equiv r]_T \in \mathcal{U})$,
  es facil ver que $(t\equiv r)$ se deduce por la regla de reemplazo, por lo que se puede demostrar $((t\equiv s) \land (s\equiv r)) \rightarrow (t\equiv r)$, y por lo tanto
  $\inft{[t\equiv s]_T}{[s\equiv r]_T} \leq^T [t\equiv r]_T$, y como $\mathcal{U}$ es un filtro, se cumple transitividad.
  
\end{itemize}


Probaremos (2). Notese que por regla de reemplazo,
\begin{equation*}
T\vdash \left( (t_{1}\equiv s_{1})\wedge (t_{2}\equiv s_{2})\wedge ...\wedge
(t_{n}\equiv s_{n})\wedge \varphi (t_{1},...,t_{n})\right) \rightarrow
\varphi (s_{1},...,s_{n})
\end{equation*}%
lo cual nos dice que%
\begin{equation*}
\lbrack (t_{1}\equiv s_{1})]_{T}\;\mathsf{i}^{T}\mathsf{\;}[(t_{2}\equiv
s_{2})]_{T}\;\mathsf{i}^{T}\mathsf{\;}...\;\mathsf{i}^{T}\mathsf{\;}%
[(t_{n}\equiv s_{n})]_{T}\;\mathsf{i}^{T}\mathsf{\;}[\varphi
(t_{1},...,t_{n})]_{T}\leq ^{T}[\varphi (s_{1},...,s_{n})]_{T}
\end{equation*}%
de lo cual se desprende que%
\begin{equation*}
\lbrack \varphi (t_{1},...,t_{n})]_{T}\in \mathcal{U}\text{ implica }%
[\varphi (s_{1},...,s_{n})]_{T}\in \mathcal{U}
\end{equation*}%
ya que $\mathcal{U}$ es un filtro. La otra implicacion es analoga.

Para probar (3) podemos tomar $\varphi =\left( f(v_{1},...,v_{n})\equiv
f(s_{1},...,s_{n})\right) $ y aplicar (2).

Definamos ahora un modelo $\mathbf{A}_{\mathcal{U}}$ de tipo $\tau $ de la
siguiente manera:

\begin{enumerate}
\item[-] Universo de $\mathbf{A}_{\mathcal{U}}=T_{c}^{\tau }/\mathrm{\bowtie 
}$

\item[-] $c^{\mathbf{A}_{\mathcal{U}}}=c/\mathrm{\bowtie }$, para cada $c\in 
\mathcal{C}$.

\item[-] $f^{\mathbf{A}_{\mathcal{U}}}(t_{1}/\mathrm{\bowtie },...,t_{n}/%
\mathrm{\bowtie })=f(t_{1},...,t_{n})/\mathrm{\bowtie }$, para cada $f\in 
\mathcal{F}_{n}$, $t_{1},...,t_{n}\in T_{c}^{\tau }\;$

\item[-] $r^{\mathbf{A}_{\mathcal{U}}}=\{(t_{1}/\mathrm{\bowtie },...,t_{n}/%
\mathrm{\bowtie }):[r(t_{1},...,t_{n})]_{T}\in \mathcal{U}\}$, para cada $%
r\in \mathcal{R}_{n}$.
\end{enumerate}

Notese que la definicion de $f^{\mathbf{A}_{\mathcal{U}}}$ es inambigua por
(3). Probaremos las siguientes propiedades basicas:

\begin{enumerate}
\item[(4)] Para cada $t=_{d}t(v_{1},...,v_{n})\in T^{\tau }$, $%
t_{1},...,t_{n}\in T_{c}^{\tau }$, tenemos que%
\begin{equation*}
t^{\mathbf{A}_{\mathcal{U}}}[t_{1}/\mathrm{\bowtie },...,t_{n}/\mathrm{%
\bowtie }]=t(t_{1},...,t_{n})/\mathrm{\bowtie }
\end{equation*}

\item[(5)] Para cada $\varphi =_{d}\varphi (v_{1},...,v_{n})\in F^{\tau }$, $%
t_{1},...,t_{n}\in T_{c}^{\tau }$, tenemos que%
\begin{equation*}
\mathbf{A}_{\mathcal{U}}\models \varphi \lbrack t_{1}/\mathrm{\bowtie }%
,...,t_{n}/\mathrm{\bowtie }]\text{ si y solo si }[\varphi
(t_{1},...,t_{n})]_{T}\in \mathcal{U}.
\end{equation*}
\end{enumerate}

La prueba de (4) es directa por induccion. Probaremos (5) por induccion en
el $k$ tal que $\varphi \in F_{k}^{\tau }$. El caso $k=0$ es dejado al
lector. Supongamos (5) vale para $\varphi \in F_{k}^{\tau }$. Sea $\varphi
=_{d}\varphi (v_{1},...,v_{n})\in F_{k+1}^{\tau }-F_{k}^{\tau }$. Hay varios
casos:

CASO $\varphi =\left( \varphi _{1}\vee \varphi _{2}\right) $.

Notese que por la Convencion Notacional 6, tenemos que $\varphi
_{i}=_{d}\varphi _{i}(v_{1},...,v_{n})$. Tenemos entonces%
\begin{equation*}
\begin{array}{c}
\mathbf{A}_{\mathcal{U}}\models \varphi \lbrack t_{1}/\mathrm{\bowtie }%
,...,t_{n}/\mathrm{\bowtie }] \\ 
\Updownarrow \\ 
\mathbf{A}_{\mathcal{U}}\models \varphi _{1}[t_{1}/\mathrm{\bowtie }%
,...,t_{n}/\mathrm{\bowtie }]\text{ o }\mathbf{A}_{\mathcal{U}}\models
\varphi _{2}[t_{1}/\mathrm{\bowtie },...,t_{n}/\mathrm{\bowtie }] \\ 
\Updownarrow \\ 
\lbrack \varphi _{1}(t_{1},...,t_{n})]_{T}\in \mathcal{U}\text{ o }[\varphi
_{2}(t_{1},...,t_{n})]_{T}\in \mathcal{U} \\ 
\Updownarrow \\ 
\lbrack \varphi _{1}(t_{1},...,t_{n})]_{T}\ \mathsf{s}^{T}\mathsf{\ }%
[\varphi _{2}(t_{1},...,t_{n})]_{T}\in \mathcal{U} \\ 
\Updownarrow \\ 
\lbrack \left( \varphi _{1}(t_{1},...,t_{n})\vee \varphi
_{2}(t_{1},...,t_{n})\right) ]_{T}\in \mathcal{U} \\ 
\Updownarrow \\ 
\lbrack \varphi (t_{1},...,t_{n})]_{T}\in \mathcal{U}.%
\end{array}%
\end{equation*}%
CASO $\varphi =\forall v\varphi _{1}$, con $v\in Var-\{v_{1},...,v_{n}\}$.
Notese que por la Convencion Notacional 6, tenemos que $\varphi
_{1}=_{d}\varphi _{1}(v_{1},...,v_{n},v)$. Tenemos entonces%
\begin{equation*}
\begin{array}{c}
\mathbf{A}_{\mathcal{U}}\models \varphi \lbrack t_{1}/\mathrm{\bowtie }%
,...,t_{n}/\mathrm{\bowtie }] \\ 
\Updownarrow \\ 
\mathbf{A}_{\mathcal{U}}\models \varphi _{1}[t_{1}/\mathrm{\bowtie }%
,...,t_{n}/\mathrm{\bowtie },t/\mathrm{\bowtie }]\text{, para todo }t\in
T_{c}^{\tau } \\ 
\Updownarrow \\ 
\lbrack \varphi _{1}(t_{1},...,t_{n},t)]_{T}\in \mathcal{U}\text{, para todo 
}t\in T_{c}^{\tau } \\ 
\Updownarrow \\ 
\lbrack \forall v\varphi _{1}(t_{1},...,t_{n},v)]_{T}\in \mathcal{U} \\ 
\Updownarrow \\ 
\lbrack \varphi (t_{1},...,t_{n})]_{T}\in \mathcal{U}.%
\end{array}%
\end{equation*}%
CASO $\varphi =\exists v\varphi _{1}$, con $v\in Var-\{v_{1},...,v_{n}\}$.
Notese que por la Convencion Notacional 6, tenemos que $\varphi
_{1}=_{d}\varphi _{1}(v_{1},...,v_{n},v)$. Tenemos entonces%
\begin{equation*}
\begin{array}{c}
\mathbf{A}_{\mathcal{U}}\models \varphi \lbrack t_{1}/\mathrm{\bowtie }%
,...,t_{n}/\mathrm{\bowtie }] \\ 
\Updownarrow \\ 
\mathbf{A}_{\mathcal{U}}\models \varphi _{1}[t_{1}/\mathrm{\bowtie }%
,...,t_{n}/\mathrm{\bowtie },t/\mathrm{\bowtie }]\text{, para algun }t\in
T_{c}^{\tau } \\ 
\Updownarrow \\ 
\lbrack \varphi _{1}(t_{1},...,t_{n},t)]_{T}\in \mathcal{U}\text{, para
algun }t\in T_{c}^{\tau } \\ 
\Updownarrow \\ 
([\varphi _{1}(t_{1},...,t_{n},t)]_{T})^{\mathsf{c}^{T}}\not\in \mathcal{U}%
\text{, para algun }t\in T_{c}^{\tau } \\ 
\Updownarrow \\ 
\lbrack \neg \varphi _{1}(t_{1},...,t_{n},t)]_{T}\not\in \mathcal{U}\text{,
para algun }t\in T_{c}^{\tau } \\ 
\Updownarrow \\ 
\lbrack \forall v\;\neg \varphi _{1}(t_{1},...,t_{n},v)]_{T}\not\in 
\mathcal{U} \\ 
\Updownarrow \\ 
([\forall v\;\neg \varphi _{1}(t_{1},...,t_{n},v)]_{T})^{\mathsf{c}^{T}}\in 
\mathcal{U} \\ 
\Updownarrow \\ 
\lbrack \neg \forall v\;\neg \varphi _{1}(t_{1},...,t_{n},v)]_{T}\in 
\mathcal{U} \\ 
\Updownarrow \\ 
\lbrack \varphi (t_{1},...,t_{n})]_{T}\in \mathcal{U}.%
\end{array}%
\end{equation*}%
Pero ahora notese que (5) en particular nos dice que para cada sentencia $%
\psi \in S^{\tau }$, $\mathbf{A}_{\mathcal{U}}\models \psi $ si y solo si $%
[\psi ]_{T}\in \mathcal{U}.$ De esta forma llegamos a que $\mathbf{A}_{%
\mathcal{U}}\models \Sigma $ y $\mathbf{A}_{\mathcal{U}}\models \neg
\varphi_0$, lo cual contradice la suposicion de que $T\models \varphi
_0.$

Ahora supongamos que $\tau $ es cualquier tipo. Sean $s_{1}$ y $s_{2}$ un
par de simbolos no pertenecientes a la lista%
\begin{equation*}
\forall \ \ \exists \ \ \neg \ \ \vee \ \ \wedge \ \ \rightarrow \ \
\leftrightarrow \ \ (\ \ )\ \ ,\ \equiv \ \ \mathsf{X}\ \ \mathit{0}\ \ 
\mathit{1}\ \ ...\ \ \mathit{9}\ \ \mathbf{0}\ \ \mathbf{1}\ \ ...\ \ 
\mathbf{9}
\end{equation*}%
y tales que ninguno ocurra en alguna palabra de $\mathcal{C}\cup \mathcal{F}%
\cup \mathcal{R}.$ Si $T\models \varphi $, entonces usando el Lema de
Coincidencia se puede ver que $(\Sigma ,(\mathcal{C}\cup
\{s_{1}s_{2}s_{1},s_{1}s_{2}s_{2}s_{1},...\},\mathcal{F},\mathcal{R}%
,a))\models \varphi $, por lo cual%
\begin{equation*}
(\Sigma ,(\mathcal{C}\cup \{s_{1}s_{2}s_{1},s_{1}s_{2}s_{2}s_{1},...\},%
\mathcal{F},\mathcal{R},a))\vdash \varphi .
\end{equation*}%
Pero por Lema de tipos parecidos, tenemos que $T\vdash \varphi .$
\end{proof}
\begin{corollary}
  Toda teoria consistente tiene un modelo
\end{corollary}
\begin{proof}
  Supongamos $(\Sigma ,\tau )$ es consistente y no tiene modelos. Entonces $%
(\Sigma ,\tau )\models \left( \varphi \wedge \lnot \varphi \right) $, con lo
cual por completitud $(\Sigma ,\tau )\vdash \left( \varphi \wedge \lnot
\varphi \right) $, lo cual es absurdo.
\end{proof}
\begin{corollary}[Teorema de Compacidad]
  Sea $\forder$ una teoria. \begin{enumerate}
    \item Si \forder es tal que $(\Sigma_0,\tau)$ tiene un modelo, para cada subconjunto finito $\Sigma_0\subseteq\Sigma$, entonces \forder tiene un modelo
    \item Si $\forder\models\varphi$, entonces hay un subconjunto finito $\Sigma_0\subseteq\Sigma$ tal que $(\Sigma_0,\tau)\models\varphi$
  \end{enumerate}
\end{corollary}
\begin{proof}
  (a) Veamos que $(\Sigma ,\tau )$ es consistente. Supongamos lo contrario, es
decir supongamos $(\Sigma ,\tau )\vdash \left( \varphi \wedge \lnot \varphi
\right) $, para alguna sentencia $\varphi $. Notese que entonces hay un
subconjunto finito $\Sigma _{0}\subseteq \Sigma $ tal que la teoria $(\Sigma
_{0},\tau )\vdash \left( \varphi \wedge \lnot \varphi \right) $ ($\Sigma
_{0} $ puede ser formado con los axiomas de $\Sigma $ usados en una prueba
formal que atestigue que $(\Sigma ,\tau )\vdash \left( \varphi \wedge \lnot
\varphi \right) $). Pero esto es absurdo ya que por hypotesis dicha teoria $%
(\Sigma _{0},\tau )$ tiene un modelo. O sea que $(\Sigma ,\tau )$ es
consistente por lo cual tiene un modelo.

(b) Si $(\Sigma ,\tau )\models \varphi $, entonces por completitud, $(\Sigma
,\tau )\vdash \varphi $. Pero entonces hay un subconjunto finito $\Sigma
_{0}\subseteq \Sigma $ tal que $(\Sigma _{0},\tau )\vdash \varphi $, es
decir tal que $(\Sigma _{0},\tau )\models \varphi $ (correccion).
\end{proof}

\section{Interpretacion semantica del algebra de Lindembaum}
\begin{definition}
  Sea $T=\forder$ una teoria. Dada $\varphi\in S^\tau$ definamos
  $$
  \text{Mod}_T(\varphi)=\{\mathbf{A}:\mathbf{A} \text{ es modelo de } T \text{ y } \mathbf{A}\models\varphi\}
  $$
\end{definition}

\begin{lemma}
  Dadas $\varphi,\psi \in S^\tau$, se tiene:\begin{enumerate}
    \item $[\varphi]_T\leq^T[\psi]_T\iff$ $\emph{Mod}_T(\varphi)\subseteq\emph{Mod}_T(\psi)$
    \item $[\varphi]_T=[\psi]_T\iff$ $\emph{Mod}_T(\varphi)=\emph{Mod}_T(\psi)$
    \item $[\varphi]_T<^T[\psi]_T\iff$ $\emph{Mod}_T(\varphi)\subsetneqq\emph{Mod}_T(\psi)$
  \end{enumerate}
\end{lemma}
\begin{proof}
  TODO
\end{proof}