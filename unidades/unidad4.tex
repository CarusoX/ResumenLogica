% !TeX root = ../resumen.tex
\begin{document}
  \section{Reticulados acotados}
  \begin{definition}
    Por un \emph{reticulado acotado} entenderemos una 5-upla \reticulAcot, tal que \reticulAlg
    es un reticulado, $0, 1 \in L$, y ademas se cumplen las siguientes propiedades
    \begin{enumerate}
      \item $\supbin{0}{x} = x$, para cada $x \in L$
      \item $\supbin{1}{x} = 1$, para cada $x \in L$
    \end{enumerate}
  \end{definition}

  \subsection{Subreticulados acotados}
  \begin{definition}
    Dados reticulados acotados \reticulAcot y \reticulAcotDef{L'}{s'}{i'}{0'}{1'} diremos que \reticulAcot
    es un $\emph{subreticulado acotado de}$ \reticulAcotDef{L'}{s'}{i'}{0'}{1'} si se dan las siguientes condiciones:
    \begin{enumerate}
      \item $L \subseteq L'$
      \item $0 = 0'$ y $1 = 1'$
      \item $s = s'\vert_{L \times L}$
      \item $i = i'\vert_{L \times L}$
    \end{enumerate}

    Sea \reticulAcot un reticulado acotado. Un conjunto $S \subseteq L$ es llamado un $\emph{subuniverso}$ de
    \reticulAcot si $0, 1 \in S$, y $S$ es cerrado bajo las operaciones $s$ e $i$.
  \end{definition}

  \subsection{Homomorfismo de reticulados acotados}
  \begin{definition}
    Sean \reticulAcot y \reticulAcotDef{L'}{s'}{i'}{0'}{1'} reticulados acotados. Una funcion \functype{F}{L}{L'}
    sera llamada un \emph{homomorfismo de} \reticulAcot en \reticulAcotDef{L'}{s'}{i'}{0'}{1'} si para todo
    $x, y \in L$ se cumple que:
    \begin{alignat*}{4}
      F(\supbin{x}{y}) &=&\ & F(x) s' F(y)\\
      F(\infbin{x}{y}) &=&\ & F(x) i' F(y)\\
      F(0) &=&\ & 0'\\
      F(1) &=&\ & 1'
    \end{alignat*}

    Un homomorfismo \reticulAcot en \reticulAcotDef{L'}{s'}{i'}{0'}{1'} sera llamado \emph{isomorfismo} cuando
    sea biyectivo y su inversa tambien sea un homomorfismo. 

    Escribiremos \functype{F}{\reticulAcot}{\reticulAcotDef{L'}{s'}{i'}{0'}{1'}} cuando $F$ sea un homomorfismo
  de \reticulAcot en \reticulAcotDef{L'}{s'}{i'}{0'}{1'}

  Escribiremos \reticulAcot $\tilde{=}$ \reticulAcotDef{L'}{s'}{i'}{0'}{1'} cuando exista
  un isomorfismo de \reticulAcot en \reticulAcotDef{L'}{s'}{i'}{0'}{1'}
  \end{definition}

  \begin{lemma}
    Si \functype{F}{\reticulAcot}{\reticulAcotDef{L'}{s'}{i'}{0'}{1'}} es un \emph{homomorfismo biyectivo},
    entonces $F$ es un isomorfismo.
  \end{lemma}
  \noproof
  \begin{lemma}
    Sean \reticulAcot y \reticulAcotDef{L'}{s'}{i'}{0'}{1'} reticulados y sea \functype{F}{\reticulAcot}{\reticulAcotDef{L'}{s'}{i'}{0'}{1'}}
    un homomorfismo. Entonces $I_F$ es un subuniverso de \reticulAcotDef{L'}{s'}{i'}{0'}{1'}. Es decir que $F$
    es tambien un homomorfismo de \reticulAcot en \reticulAcotDef{I_F}{\textbf{s}'\vert_{I_F \times I_F}}{\textbf{i}'\vert_{I_F \times I_F}}{0'}{1'}
  \end{lemma}
  \noproof

  \subsection{Congruencias de reticulados acotados}
  \begin{definition}
    Sea \reticulAcot un reticulado acotado. Una \emph{congruencia sobre} \reticulAcot sera una
    relacion de equivalencia $\theta$ la cual sera una congruencia sobre \reticulAlg. Tenemos definidas sobre
    $L/\theta$ dos operaciones binarias $\tilde{\textbf{s}}$ y $\tilde{\text{i}}$
    \begin{alignat*}{2}
      \supbincongr{x/\theta}{y/\theta} &=& (\supbin{x}{y})/\theta\\
      \infbincongr{x/\theta}{y/\theta} &=& (\infbin{x}{y})/\theta
    \end{alignat*}

    La 5-upla \reticulAcotDef{L/\theta}{\tilde{\textbf{s}}}{\tilde{\textbf{i}}}{0/\theta}{1/\theta} es llamada
    \emph{cociente de} \reticulAcot sobre $\theta$ y la denotaremos con $\reticulAcot/\theta$
  \end{definition}

  \begin{lemma}
    Sea \reticulAcot un reticulado acotado y $\theta$ una congruencia sobre \reticulAcot.
    \begin{enumerate}
      \item \reticulAcotDef{L/\theta}{\tilde{\textbf{s}}}{\tilde{\textbf{i}}}{0/\theta}{1/\theta} es un reticulado acotado
      \item $\pi_\theta$ es un homomorfismo de \reticulAcot en \reticulAcotDef{L/\theta}{\tilde{\textbf{s}}}{\tilde{\textbf{i}}}{0/\theta}{1/\theta} cuyo nucleo es $\theta$
    \end{enumerate}
  \end{lemma}

  \begin{proof}
    (1) Cuando hablemos de $z, w \in L/\theta$, automaticamente tendremos definidos $x, y \in L$ tal que $x/\theta = z$ y $y/\theta = w$. Demostraremos una a una las propiedades que se deben cumplir:
    \begin{itemize}
      \item $\supbincongr{z}{z} = \infbincongr{z}{z} = z$, cualesquiera sea $z \in L/\theta$, pues $\supbin{x}{x} = \infbin{x}{x} = x$
      \item $\supbincongr{z}{w} = \supbincongr{w}{z}$, cualesquiera sean $z, w \in L/\theta$, pues $\supbin{x}{y} = \supbin{y}{x}$
      \item $\infbincongr{z}{w} = \infbincongr{w}{z}$, cualesquiera sean $z, w \in L/\theta$, pues $\infbin{x}{y} = \infbin{y}{x}$
      \item ... Faciles de demostrar ...
      \item $\supbincongr{0/\theta}{z} = z$, para cada $z \in L/\theta$, pues $\supbin{0}{x} = x$
      \item $\supbincongr{1/\theta}{z} = 1$, para cada $z \in L/\theta$, pues $\supbin{1}{x} = 1$
    \end{itemize}

    (2) Es directo de su analogo para reticulados ternas. Capaz en un futuro lo hago
  \end{proof}

  \begin{lemma}
    Si \functype{F}{\reticulAcot}{\reticulAcotDef{L'}{s'}{i'}{0'}{1'}} es un homomorfismo de
    reticulados acotados, entonces ker $F$ es una congruencia sobre \reticulAcot
  \end{lemma}
  \begin{proof}
    Es directo de su analogo para reticulados ternas. Capaz en un futuro lo hago
  \end{proof}

  \section{Reticulados complementados}
  \begin{definition}
    Sea \reticulAcot un reticulado acotado. Dado $a \in L$, diremos que $a$ es \emph{complementado} cuando 
    exista un elemento $b \in L$ (llamado \emph{complemento de a}) tal que:
    \begin{alignat*}{4}
      \supbin{a}{b} &=&\ & 1\\
      \infbin{a}{b} &=&\ & 0
    \end{alignat*}
  \end{definition}

  \begin{definition}
    Entonderemos por \emph{reticulado complementado} a una 6-upla \reticulCompl tal que \reticulAcot es un
    reticulado acotado y ${}^c$ es una operacion unaria sobre $L$ tal que:
    \begin{enumerate}
      \item \supbin{x}{x^c} = 1, para cada $x \in L$
      \item \infbin{x}{x^c} = 0, para cada $x \in L$
    \end{enumerate}
  \end{definition}

  \subsection{Subreticulados complementados}
  \begin{definition}
    Dados reticulados complementados \reticulCompl y \reticulComplDef{L'}{s'}{i'}{{}^{c'}}{0'}{1'} diremos que
    \reticulCompl es un \emph{subreticulado complementado de} \reticulComplDef{L'}{s'}{i'}{{}^{c'}}{0'}{1'} si se
    dan las siguientes condiciones:
    \begin{enumerate}
      \item $L \subseteq L'$
      \item $0 = 0'$ y $1 = 1'$
      \item $s = s'\vert_{L \times L}$
      \item $i = i'\vert_{L \times L}$
      \item ${}^c = {}^{c'}\vert_L$
    \end{enumerate}

    Sea \reticulCompl un reticulado complementado. Un conjunto $S \subseteq L$ es llamado un $\emph{subuniverso}$ de
    \reticulCompl si $0, 1 \in S$, y $S$ es cerrado bajo las operaciones $s$, $i$ y ${}^c$.
  \end{definition}

  \subsection{Homomorfismo de reticulados complementados}

  \begin{definition}
    Sean \reticulCompl y \reticulComplDef{L'}{s'}{i'}{{}^{c'}}{0'}{1'} reticulados complementados. Una funcion
    \functype{F}{L}{L'} sera llamada un \emph{homomorfismo de} \reticulCompl en \reticulComplDef{L'}{s'}{i'}{{}^{c'}}{0'}{1'}
    si para todo $x, y \in L$ se cumple que:
    \begin{alignat*}{4}
      F(\supbin{x}{y}) &=&\ & F(x) s' F(y)\\
      F(\infbin{x}{y}) &=&\ & F(x) i' F(y)\\
      F(x^c) &=&\ & F(x)^{c'}\\
      F(0) &=&\ & 0'\\
      F(1) &=&\ & 1'
    \end{alignat*}

    Un homomorfismo \reticulCompl en \reticulComplDef{L'}{s'}{i'}{{}^{c'}}{0'}{1'} sera llamado \emph{isomorfismo} cuando
    sea biyectivo y su inversa tambien sea un homomorfismo. 

    Escribiremos \functype{F}{\reticulCompl}{\reticulComplDef{L'}{s'}{i'}{{}^{c'}}{0'}{1'}} cuando $F$ sea un homomorfismo de\\
    \reticulCompl en \reticulComplDef{L'}{s'}{i'}{{}^{c'}}{0'}{1'}

    Escribiremos \reticulCompl $\tilde{=}$ \reticulComplDef{L'}{s'}{i'}{{}^{c'}}{0'}{1'} cuando exista
    un isomorfismo de \reticulCompl en \reticulComplDef{L'}{s'}{i'}{{}^{c'}}{0'}{1'}
  \end{definition}

  \begin{lemma}
    Si \functype{F}{\reticulCompl}{\reticulComplDef{L'}{s'}{i'}{{}^{c'}}{0'}{1'}} es un \emph{homomorfismo biyectivo},
    entonces $F$ es un isomorfismo.
  \end{lemma}
  \noproof
  \begin{lemma}
    Sean \reticulCompl y \reticulComplDef{L'}{s'}{i'}{{}^{c'}}{0'}{1'} reticulados y sea \functype{F}{\reticulCompl}{\reticulComplDef{L'}{s'}{i'}{{}^{c'}}{0'}{1'}}
    un homomorfismo. Entonces $I_F$ es un subuniverso de \reticulComplDef{L'}{s'}{i'}{{}^{c'}}{0'}{1'}. Es decir que $F$
    es tambien un homomorfismo de \reticulCompl en \reticulComplDef{I_F}{\textbf{s}'\vert_{I_F \times I_F}}{\textbf{i}'\vert_{I_F \times I_F}}{{}^{c'}\vert_{I_F}}{0'}{1'}
  \end{lemma}
  \noproof

  \subsection{Congruencias de reticulados complementados}
  \begin{definition}
    Sea \reticulCompl un reticulado complementado. Una \emph{congruencia sobre} \reticulCompl sera una
    relacion de equivalencia $\theta$ sobre $L$ la cual cumpla: 
    \begin{enumerate}
      \item $\theta$ es una congruencia sobre \reticulAcot
      \item $x/\theta = y/\theta$ implica $x^c/\theta = y^c/\theta$
    \end{enumerate}

    Las condiciones anteriores nos permiten definir sobre $L/\theta$ dos operaciones binarias $\tilde{\textbf{s}}$ y $\tilde{\text{i}}$
    y una operacion binaria ${}^{\tilde{c}}$ de la siguiente manera:
    \begin{alignat*}{3}
      \supbincongr{x/\theta}{y/\theta} &=&\ & (\supbin{x}{y})/\theta\\
      \infbincongr{x/\theta}{y/\theta} &=&\ & (\infbin{x}{y})/\theta\\
      (x/\theta)^{\tilde{c}} &=&\ & x^c/\theta
    \end{alignat*}

    La 6-upla \reticulComplDef{L/\theta}{\tilde{\textbf{s}}}{\tilde{\textbf{i}}}{{}^{\tilde{c}}}{0/\theta}{1/\theta} es llamada
    \emph{cociente de} \reticulCompl sobre $\theta$ y la denotaremos con $\reticulCompl/\theta$
  \end{definition}

  \begin{lemma}
    Sea \reticulCompl un reticulado complementado y $\theta$ una congruencia sobre \reticulCompl.
    \begin{enumerate}
      \item \reticulComplDef{L/\theta}{\tilde{\textbf{s}}}{\tilde{\textbf{i}}}{{}^{\tilde{c}}}{0/\theta}{1/\theta} es un reticulado complementado
      \item $\pi_\theta$ es un homomorfismo de \reticulCompl en \reticulComplDef{L/\theta}{\tilde{\textbf{s}}}{\tilde{\textbf{i}}}{{}^{\tilde{c}}}{0/\theta}{1/\theta} cuyo nucleo es $\theta$
    \end{enumerate}
  \end{lemma}
  \begin{proof}
    (1) Por un lema anterior, ya sabemos que \reticulAcotDef{L/\theta}{\tilde{\textbf{s}}}{\tilde{\textbf{i}}}{0/\theta}{1/\theta} es un reticulado acotado.
    Osea solo nos falta ver que \reticulComplDef{L/\theta}{\tilde{\textbf{s}}}{\tilde{\textbf{i}}}{{}^{\tilde{c}}}{0/\theta}{1/\theta} satisface las propiedades de reticulado complementado.
    Sea $x/\theta \in L/\theta$. Sabemos que $\supbin{x}{x^c} = 1$ y por lo tanto $\supbincongr{x/\theta}{x^c/\theta} = (\supbin{x}{x^c})/\theta = 1/\theta$.
    Similarmente, sabemos que $\infbin{x}{x^c} = 0$ y por lo tanto $\infbincongr{x}{x^c} = 0/\theta$

    (2) Por lema anterior tenemos que $\pi_\theta$ es un homomorfismo de \reticulAcot en \reticulAcotDef{L/\theta}{\tilde{\textbf{s}}}{\tilde{\textbf{i}}}{0/\theta}{1/\theta} cuyo nucleo es $\theta$.
    Notese que por definicion de ${}^{\tilde{c}}$ tenemos que $x^c/\theta = (x/\theta)^{\tilde{c}}$, y por lo tanto $\pi_\theta(x^c) = (\pi_\theta(x))^{\tilde{c}}$,
    cualquiera sea $x \in L$.

  \end{proof}

  \begin{lemma}
    Si \functype{F}{\reticulCompl}{\reticulComplDef{L'}{s'}{i'}{{}^{c'}}{0'}{1'}} es un homomorfismo de
    reticulados complementados, entonces ker $F$ es una congruencia sobre \reticulCompl
  \end{lemma}
  \noproof
\end{document}