% !TeX root = ../resumen.tex
\begin{document}
\section{Version algebraica del concepto de reticulado}
\begin{definition}
  Una terna \reticulAlg, donde $L$ es un conjunto y \textbf{s}, \textbf{i} son dos operaciones
  binarias sobre $L$ sera llamada \emph{reticulado} cuando cumpla:
  \begin{enumerate}
    \item $\supbin{x}{x} = \infbin{x}{x} = x$, cualesquiera sea $x \in L$
    \item $\supbin{x}{y} = \supbin{y}{x}$, cualesquiera sean $x,y \in L$
    \item $\infbin{x}{y} = \infbin{y}{x}$, cualesquiera sean $x,y \in L$
    \item $\supbin{(\supbin{x}{y})}{z} = \supbin{x}{(\supbin{y}{z})}$, cualesquiera sean $x,y,z \in L$
    \item $\infbin{(\infbin{x}{y})}{z} = \infbin{x}{(\infbin{y}{z})}$, cualesquiera sean $x,y,z \in L$
    \item $\supbin{x}{(\infbin{x}{y})} = x$, cualesquiera sean $x,y \in L$
    \item $\infbin{x}{(\supbin{x}{y})} = x$, cualesquiera sean $x,y \in L$
  \end{enumerate}

  En tal caso que \reticulAlg sea un reticulado, diremos que $L$ es el \emph{universo} del reticulado.
\end{definition}

\begin{theorem}
  Sea \reticulAlg un reticulado. La relacion binaria definida por:
  $$
  x \leq y \iff \supbin{x}{y} = y
  $$
  es un orden parcial sobre L para el cual se cumple que:
  \begin{alignat*}{2}
    \sup&(\{x, y\}) &\ = \supbin{x}{y}\\
    \inf&(\{x, y\}) &\ = \infbin{x}{y}
  \end{alignat*}
  cualesquiera sean $x, y \in L$
\end{theorem}
\begin{definition}
  Dados reticulados \reticulAlg y \reticulAlgDef{L'}{\textbf{s}'}{\textbf{i}'} diremos que \reticulAlg
  es un \emph{subreticulado de} \reticulAlgDef{L'}{\textbf{s}'}{\textbf{i}'} si se dan las siguientes condiciones:
  \begin{enumerate}
    \item $L \subseteq L'$
    \item $\textbf{s} = \textbf{s}'\vert_{L\times L}$
    \item $\textbf{i} = \textbf{i}'\vert_{L\times L}$
  \end{enumerate}
\end{definition}
\begin{definition}
  Sea \reticulAlg un reticulado. Un conjunto $S \subseteq L$ es llamado \emph{subuniverso de} \reticulAlg
  si es no vacio y cerrado bajo las operaciones \textbf{s} y \textbf{i}
\end{definition}
\begin{remark}
  Sea \reticulAlg un reticulado. $S$ es subuniverso de \reticulAlg $\iff$ \reticulAlgDef{S}{\textbf{s}\vert_{S \times S}}{\textbf{i}\vert_{S \times S}}
  es un subreticulado de \reticulAlg
\end{remark}
\begin{definition}
  Sean \reticulAlg y \reticulAlgDef{L'}{\textbf{s}'}{\textbf{i}'} reticulados. Una funcion \functype{F}{L}{L'}
  sera llamada un \emph{homomorfismo} de \reticulAlg \emph{en} \reticulAlgDef{L'}{\textbf{s}'}{\textbf{i}'} si
  para todo $x, y \in L$ se cumple que:
  \begin{alignat*}{2}
    F(\supbin{x}{y}) &\ = F(x)\textbf{ s' }F(y)\\
    F(\infbin{x}{y}) &\ = F(x)\textbf{ i' }F(y)\\    
  \end{alignat*}

  Un homomorfismo de \reticulAlg en \reticulAlgDef{L'}{\textbf{s}'}{\textbf{i}'} sera llamada \emph{isomorfismo}
  de \reticulAlg \emph{en} \reticulAlgDef{L'}{\textbf{s}'}{\textbf{i}'} cuando sea biyectivo, y su inversa
  sea tambien un homomorfismo.

  Escribiremos \functype{F}{\reticulAlg}{\reticulAlgDef{L'}{\textbf{s}'}{\textbf{i}'}} cuando $F$ sea un homomorfismo
  de \reticulAlg en \reticulAlgDef{L'}{\textbf{s}'}{\textbf{i}'}

  Escribiremos \reticulAlg $\overset{\sim}{=}$ \reticulAlgDef{L'}{\textbf{s}'}{\textbf{i}'} cuando exista
  un isomorfismo de \reticulAlg en \reticulAlgDef{L'}{\textbf{s}'}{\textbf{i}'}
\end{definition}

\begin{lemma}
  Si \functype{F}{\reticulAlg}{\reticulAlgDef{L'}{\textbf{s}'}{\textbf{i}'}} es un homomorfismo biyectivo,
  entonces $F$ es un isomorfismo
\end{lemma}
\begin{proof}
  TODO
\end{proof}
\begin{lemma}
  Sean \reticulAlg y \reticulAlgDef{L'}{\textbf{s}'}{\textbf{i}'} reticulados y sea \functype{F}{\reticulAlg}{\reticulAlgDef{L'}{\textbf{s}'}{\textbf{i}'}}
  un homomorfismo. Entonces $I_F$ es un subuniverso de \reticulAlgDef{L'}{\textbf{s}'}{\textbf{i}'}. Es decir que $F$
  es tambien un homomorfismo de \reticulAlg en \reticulAlgDef{I_F}{\textbf{s}'\vert_{I_F \times I_F}}{\textbf{i}'\vert_{I_F \times I_F}}
\end{lemma}
\begin{proof}
  TODO
\end{proof}
\begin{lemma}
  Sean \reticulAlg y \reticulAlgDef{L'}{\textbf{s}'}{\textbf{i}'} reticulados y sean \posetdef{L}{\leq} y \posetdef{L'}{\leq'}
  los posets asociados. Sea \functype{F}{L}{L'} una funcion. Entonces $F$ es un isomorfismo de
  \reticulAlg en \reticulAlgDef{L'}{\textbf{s}'}{\textbf{i}'} $\iff$ $F$ es un isomorfismo de \posetdef{L}{\leq} en \posetdef{L'}{\leq'}
\end{lemma}
\begin{proof}
  TODO
\end{proof}
\begin{definition}
  Sea \reticulAlg un reticulado. Una \emph{congruencia sobre} \reticulAlg sera una relacion de equivalencia
  $\theta$ la cual cumpla:
  $$
  x\theta x' \text{ y } y\theta y' \Rightarrow (\supbin{x}{y})\theta(\supbin{x'}{y'}) \text{ y } (\infbin{x}{y})\theta(\infbin{x'}{y'})
  $$

  Gracias a tal propiedad podemos definir sobre $L/\theta$ dos operaciones binarias $\overset{\sim}{\textbf{s}}$ y $\overset{\sim}{\text{i}}$
  \begin{alignat*}{2}
    \supbincongr{x/\theta}{y/\theta} &=& (\supbin{x}{y})/\theta\\
    \infbincongr{x/\theta}{y/\theta} &=& (\infbin{x}{y})/\theta
  \end{alignat*}
\end{definition}

\begin{definition}
  La terna \reticulAlgDef{L/\theta}{\overset{\sim}{\textbf{s}}}{\overset{\sim}{\textbf{i}}} es llamada \emph{cociente de} \reticulAlg sobre $\theta$,
  y la denotaremos con $\reticulAlg/\theta$
\end{definition}

\begin{lemma}
  \reticulAlgDef{L/\theta}{\overset{\sim}{\textbf{s}}}{\overset{\sim}{\textbf{i}}} es un reticulado. El orden
  parcial $\overset{\sim}{\leq}$ asociado a este reticulado cumple:
  $$
  x/\theta \overset{\sim}{\leq} y/\theta \iff y\theta(\supbin{x}{y})
  $$
\end{lemma}

\begin{proof}
  TODO
\end{proof}

\begin{lemma}
  Si \functype{F}{\reticulAlg}{\reticulAlgDef{L'}{s'}{i'}} es un \emph{homomorfismo}, entonces \text{ker }F es una \emph{congruencia sobre} \reticulAlg
\end{lemma}
\begin{proof}
  TODO
\end{proof}

\begin{lemma}
  Sea \reticulAlg un reticulado y sea $\theta$ una congruencia sobre \reticulAlg. Entonces $\pi_\theta$ es un 
  homomorfismo de \reticulAlg en \reticulAlgDef{L/\theta}{\overset{\sim}{\textbf{s}}}{\overset{\sim}{\textbf{i}}}.
  Ademas ker $\pi_\theta$ = $\theta$.
\end{lemma}
\begin{proof}
  TODO
\end{proof}

\end{document}