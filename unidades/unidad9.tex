% !TeX root = ../resumen.tex

\section{Teorias de primer orden}
\begin{definition}
  Una \emph{teoria de primer orden} sera un par \forder, donde $\tau$ es un tipo
  y $\Sigma$ es un conjunto de sentencias de tipo $\tau$. Los elementos de $\Sigma$ seran llamados
  \emph{axiomas propios} de \forder. Un $\emph{modelo}$ de \forder sera una estructura de tipo $\tau$
  la cual satisfaga todos los axiomas propios de \forder.
\end{definition}

\section{Definicion del concepto de prueba}

\subsection{Reglas}
\begin{definition}
  Definiremos una serie de conjuntos los cuales poseen informacion deductiva basica.
  Sea $T_c^\tau$ el conjunto de los terminos cerrados de tipo $\tau$.
  Sean
  \begin{alignat}{3}
    &Partic^\tau &\ =&\ \{(\forall v\varphi(v),\varphi(t)): \varphi =_d\varphi(v)\in F^\tau \text{ y } t\in T_c^\tau\}\\
    &Exist^\tau &\ =&\ \{(\varphi(t),\exists v\varphi(v)): \varphi =_d\varphi(v)\in F^\tau \text{ y } t\in T_c^\tau\}\\    
    &Evoc^\tau &\ =&\ \{(\varphi,\varphi): \varphi\in S^\tau\}\\    
    &Absur^\tau &\ =&\ \{((\neg\varphi\rightarrow(\psi\land\neg\psi)),\varphi): \varphi,\psi \in S^\tau\}\cup\{((\varphi\rightarrow(\psi\land\neg\psi)),\neg\varphi): \varphi,\psi \in S^\tau\}\\
    &ConjElim^\tau &\ =&\ \{((\varphi\land\psi),\varphi)):\varphi,\psi\in S^\tau\}\cup\{((\varphi\land\psi),\psi)):\varphi,\psi\in S^\tau\}\\
    &EquivElim^\tau &\ =&\ \{((\varphi\leftrightarrow\psi),(\varphi\rightarrow\psi)):\varphi,\psi\in S^\tau\}\cup\{((\varphi\leftrightarrow\psi),(\psi\rightarrow\varphi)):\varphi,\psi\in S^\tau\}\\
    &DisjInt^\tau &\ =&\ \{(\varphi, (\varphi\lor\psi)):\varphi,\psi\in S^\tau\}\cup\{(\psi, (\varphi\lor\psi)):\varphi,\psi\in S^\tau\}
  \end{alignat}

  Diremos que $\varphi$ se deduce de $\psi$ por la regla de \emph{particularizacion} (resp. \emph{existencia, evocacion,
  absurdo, conjuncion-eliminacion, equivalencia-eliminacion, disjuncion-introduccion}), con respecto a $\tau$ para expresar que
  $(\psi,\varphi)\in Partic^\tau$ (resp. $Exist^\tau,Evoc^\tau,Absur^\tau,ConjElim^\tau,EquivElim^\tau,DisjInt^\tau$).

  Sea
  $$
  Commut^\tau = Commut1^\tau \cup Commut2^\tau
  $$
  donde 
  \begin{alignat*}{3}
    &Commut1^\tau &\ =&\ \{((t\equiv s), (s\equiv t)):s,t\in T_c^\tau\}\\
    &Commut2^\tau &\ =&\ \{((\varphi\leftrightarrow\psi), (\psi\leftrightarrow\varphi)):\varphi,\psi\in S^\tau\}    
  \end{alignat*}
  Diremos que $\varphi$ se deduce de $\psi$ por regla de \emph{commutatividad}, con respecto a $\tau$
  para expresar que $(\psi,\varphi)\in Commut^\tau$

  Sean
  \begin{alignat*}{3}
    &ModPon^\tau &\ =&\ \{(\varphi,(\varphi\rightarrow\psi),\psi):\varphi,\psi\in S^\tau\}\\
    &ConjInt^\tau &\ =&\ \{(\varphi,\psi,(\varphi\land\psi)):\varphi,\psi\in S^\tau\}\\
    &EquivInt^\tau &\ =&\ \{((\varphi\rightarrow\psi),(\psi\rightarrow\varphi),(\varphi\leftrightarrow\psi)):\varphi,\psi\in S^\tau)\}\\
    &DisjElim^\tau &\ =&\ \{(\neg\varphi,(\varphi\lor\psi),\psi):\varphi,\psi\in S^\tau\}\cup\{(\neg\psi,(\varphi\lor\psi),\varphi):\varphi,\psi\in S^\tau\}
  \end{alignat*}

  Diremos que $\varphi$ se deduce de $\psi_1$ y $\psi_2$ por la regla de \emph{Modus Ponens} (resp.
  \emph{conjuncion-introduccion, equivalencia-introduccion, disjuncion-eliminacion}), con respecto a $\tau$ para
  expresar que $(\psi_1,\psi_2,\varphi) \in ModPon^\tau$ (resp. $ConjInt^\tau,EquivInt^\tau,DisjElim^\tau$).

  Sea
  $$
  DivPorCas^\tau = \{((\varphi_1\lor\varphi_2),(\varphi_1\rightarrow\psi),(\varphi_2\rightarrow\psi),\psi):\varphi_1,\varphi_2,\psi\in S^\tau\}
  $$

  Diremos que $\varphi$ se deduce de $\psi_1,\psi_2,\psi_3$ por la regla de \emph{division por casos} con respecto a $\tau$
  para expresar que $(\psi_1,\psi_2,\psi_3,\varphi) \in DivPorCas^\tau$.

  Sea
  $$
  Reemp^\tau = Reemp1^\tau \cup Reemp2^\tau
  $$
  donde 
  \begin{alignat*}{5}
    &Reemp1^\tau &\ =&\ \{((t\equiv s), \gamma,\hat\gamma):s,t\in T_c^\tau, \gamma\in S^\tau, \hat\gamma = \text{ resultado de reemplazar en $\gamma$ una ocurrencia de t por s}\}\\
    &Reemp2^\tau &\ =&\ \{(\forall v_1\dots v_n(\varphi\leftrightarrow\psi),\gamma,\hat\gamma):\varphi,\psi\in F^\tau, Li(\varphi)=Li(\psi)=\{\succession{v}{1}{n}, n\geq 0, \gamma \in S^\tau,&&\\
    & &&\qquad\qquad\qquad\qquad \hat\gamma = \text{ resultado de reemplazar en $\gamma$ una ocurrencia de $\varphi$ por $\psi$}\}\}
  \end{alignat*}

  Diremos que $\varphi$ se deduce de $\psi_1, \psi_2$ por la regla de \emph{reemplazo}, con respecto a $\tau$  para expresar que $(\psi_1,\psi_2,\varphi)\in Reemp^\tau$.

  Sea
  $$
  Trans^\tau = Trans1^\tau \cup Trans2^\tau \cup Trans3^\tau
  $$
  donde 
  \begin{alignat*}{5}
    &Trans1^\tau &\ =&\ \{((t\equiv s),(s\equiv u),(t\equiv u)):t,s,u\in T_c^\tau\}\\
    &Trans2^\tau &\ =&\ \{((\varphi\rightarrow\psi),(\psi\rightarrow\varPhi),(\varphi\rightarrow\varPhi)):\varphi,\psi,\varPhi\in S^\tau\}\\
    &Trans3^\tau &\ =&\ \{((\varphi\leftrightarrow\psi),(\psi\leftrightarrow\varPhi),(\varphi\leftrightarrow\varPhi)):\varphi,\psi,\varPhi\in S^\tau\}
  \end{alignat*}

  Diremos que $\varphi$ se deduce de $\psi_1,\psi_2$ por la regla de \emph{transitividad}, con respecto a $\tau$ para expresar que 
  $(\psi_1,\psi_2,\varphi) \in Trans^\tau$.

  Sea 
  \begin{alignat*}{3}
    Generaliz^\tau=\{(\psi,\forall v\hat\psi): \psi\in S^\tau, v \text{ no ocurre en $\psi$ y existe $c \in \mathcal{C}$ tal que $c$ ocurre en $\psi$ y} \\
    \text{$\hat\psi$ = resultado de reemplazar en $\psi$ cada ocurrencia de $c$ por $v$}\}
  \end{alignat*}

  \begin{lemma}
    Si $(\varphi_1,\varphi_2) \in Generaliz^\tau$, entonces el nombre de constante $c$ del cual habla la definicion de $Generaliz^\tau$ esta univocamente determinado
    por el par $(\varphi_1,\varphi_2)$
  \end{lemma}
  \begin{proof}
    Notese que $c$ es el unico nombre de constante que ocurre en $\varphi_1$ y no ocurre en $\varphi_2$.
  \end{proof}

  Escribiremos $(\varphi_1,\varphi_2)\in Generaliz^\tau$ \emph{via c} para expresar que $(\varphi_1,\varphi_2) \in Generaliz^\tau$ y que $c$
  es el unico nombre de constante que ocurre en $\varphi_1$ y no ocurre en $\varphi_2$. Diremos que $\varphi_2$ se deduce de $\varphi_1$ por la regla de  \emph{generalizacion con nombre de constante c},
  con respecto a $\tau$, para expresar que $(\varphi_1,\varphi_2)\in Generaliz^\tau$ \emph{via c}.

  Sea 
  $$
  Elec^\tau = \{(\exists v\varphi(v), \varphi(e)):\varphi =_d \varphi(v)\in F^\tau, Li(\varphi) =\{v\}, \text{ y } e\in \mathcal{C} \text{ no ocurre en }\varphi\}
  $$

  \begin{lemma}
    Si $(\varphi_1,\varphi_2) \in Elec^\tau$, entonces el nombre de constante $e$ del cual habla la definicion de $Elec^\tau$
    esta univocamente determinado por el par $(\varphi_1,\varphi_2)$
  \end{lemma}
  \begin{proof}
    Notese que $e$ es el unico nombre de constante que ocurre en $\varphi_2$ pero no en $\varphi_1$. Esto es porque en la definicion, $Li(\varphi) = \{v\}$,
    por lo tanto sabemos que $e$ va a ocurrir en $\varphi_2$.
  \end{proof}

  Escribiremos $(\varphi_1,\varphi_2)\in Elec^\tau$ \emph{via e} para expresar que $(\varphi_1,\varphi_2) \in Elec^\tau$ y que $e$
  es el unico nombre de constante que ocurre en $\varphi_2$ y no ocurre en $\varphi_1$. Diremos que $\varphi_2$ se deduce de $\varphi_1$ por la regla de  \emph{eleccion con nombre de constante e},
  con respecto a $\tau$, para expresar que $(\varphi_1,\varphi_2)\in Elec^\tau$ \emph{via e}.
\end{definition}

\begin{lemma}
  Sea $\tau$ un tipo. Todas las reglas excepto las reglas de eleccion y generalizacion son universales en el sentido que si $\varphi$ se deduce de $\succession{\psi}{1}{k}$ por alguna de estas reglas,
  entonces $((\succession[\land]{\psi}{1}{k})\rightarrow\varphi)$ es una sentencia valida.
\end{lemma}
\begin{proof}
  Veamos que la regla de existencia es universal. Por definicion, un par de $Exist^{\tau }$
  es siempre de la forma $(\varphi (t),\exists v\varphi (v))$,
  con $\varphi =_{d}\varphi (v)$ y $t\in T_{c}^{\tau }$. Sea $\mathbf{A}$ una
  estructura de tipo $\tau $ tal que $\mathbf{A}\models \varphi (t)$. Sea $t^{\mathbf{A}}$
  el valor que toma $t$ en $\mathbf{A}$. Por el Lema de reemplazo
  tenemos que $\mathbf{A}\models \varphi \left[ t^{\mathbf{A}}\right] $, por
  lo cual tenemos que $\mathbf{A}\models \exists v\varphi (v)$.
  
  Veamos que la regla modus ponens es universal. Por definicion, una tripla de $ModPon^\tau$
  es siempre de la forma $(\varphi,(\varphi\rightarrow\psi), \psi)$, con $\varphi,\psi \in S^\tau$.
  Sea $\mathbf{A}$ una estructura de tipo $\tau$ tal que $\mathbf{A}\models\varphi$ y $\mathbf{A}\models(\varphi\rightarrow\psi)$.
  Como $\mathbf{A}\models(\varphi\rightarrow\psi)$, tenemos que $\mathbf{A}\not\models\varphi$ o $\mathbf{A}\models\psi$. Como 
  la primera opcion no es factible, deducimos que $\mathbf{A}\models\psi$.
\end{proof}

\subsection{Axiomas logicos}
\begin{definition}
  Llamaremos \emph{axiomas logicos de tipo $\tau$} a todas las sentencias de alguna de las siguientes formas:
  \begin{itemize}
    \item $(\varphi\leftrightarrow\varphi)$
    \item $(t\equiv t)$
    \item $(\varphi\lor\neg\varphi)$
    \item $(\varphi\leftrightarrow\neg\neg\varphi)$
    \item $(\neg\forall v\psi\leftrightarrow\exists v\neg\psi)$
    \item $(\neg\exists v\psi\leftrightarrow\forall v\neg\psi)$
  \end{itemize}
  donde $t \in T_c^\tau, \varphi\in S^\tau, \psi\in F^\tau, v\in Var$ y $Li(\psi) \subseteq \{v\}$. Con $AxLog^\tau$ denotaremos el conjunto
  $$
  \{\varphi \in S^\tau : \varphi \text{ es un axioma logico de tipo } \tau\}
  $$
\end{definition}

\subsection{Justificaciones}
\begin{definition}
  Llamaremos \emph{numerales a lo siguientes simbolos}:
  $$
  0\quad1\quad2\quad3\quad4\quad5\quad6\quad7\quad8\quad9
  $$
  Usaremos $Num$ para denotar al conjunto de numerales. Notese que $Num\cap\omega=\emptyset$. Sea \functype{S}{Num^*}{Num^*} definida de la siguiente manera:
  \begin{alignat*}{3}
    &S(\varepsilon) &\ =&\  1\\
    &S(\alpha0) &\ =&\ \alpha1\\
    &S(\alpha1) &\ =&\ \alpha2\\
    &S(\alpha2) &\ =&\ \alpha3\\
    &S(\alpha3) &\ =&\ \alpha4\\
    &S(\alpha4) &\ =&\ \alpha5\\
    &S(\alpha5) &\ =&\ \alpha6\\
    &S(\alpha6) &\ =&\ \alpha7\\
    &S(\alpha7) &\ =&\ \alpha8\\
    &S(\alpha8) &\ =&\ \alpha9\\
    &S(\alpha9) &\ =&\ S(\alpha)0
  \end{alignat*}

  Definamos \functype{\overline{\text{ }}}{\omega}{Num^*} de la siguiente manera:
  \begin{alignat*}{3}
    &\overline{0} &\ =&\  \varepsilon\\
    &\overline{n+1} &\ =&\ S(\overline{n})\\    
  \end{alignat*}

  Sea $Nombres_1$ el conjunto formado por las siguientes palabras:
  \begin{center}
    EXISTENCIA\\
    COMMUTATIVIDAD\\
    PARTICULARIZACION\\
    ABSURDO\\
    EVOCACION\\
    CONJUNCIONELIMINACION\\
    EQUIVALENCIAELIMINACION\\
    DISJUNCIONINTRODUCCION\\
    ELECCION\\
    GENERALIZACION\\
  \end{center}

  Sea $Nombres_2$ el conjunto formado por las siguientes palabras:
  \begin{center}
    MODUSPONENS\\
    TRANSITIVIDAD\\
    CONJUNCIONINTRODUCCION\\
    EQUIVALENCIAINTRODUCCION\\
    DISJUNCIONELIMINACION\\
    REEMPLAZO\\
  \end{center}

  Una \emph{justificacion basica} es una palabras perteneciente a la union de los siguientes conjuntos de palabras:
  \begin{center}
    \{CONCLUSION, AXIOMAPROPIO, AXIOMALOGICO\}\\
    $\{\alpha(\overline{k}):k\in \mathbf{N} \text{ y } \alpha \in Nombres_1\}$\\
    $\{\alpha(\overline{j},\overline{k}):j,k\in \mathbf{N} \text{ y } \alpha \in Nombres_2\}$\\
    \{DIVISIONPORCASOS($\overline{j},\overline{k},\overline{l}):j,k,l\in\mathbf{N}$\}
  \end{center}
  Usaremos $JustBas$ para denotar al conjunto formado por todas las justificaciones basicas. Una \emph{justificacion} es una
  palabra que ya sea es una justificacion basica o pertenece a la union de los siguientes conjuntos de palabras:
  \begin{center}
    \{HIPOTESIS$\overline{k}:k\in\mathbf{N}$\}\\
    \{TESIS$\overline{j}\alpha:j\in\mathbf{N}\text{ y }\alpha\in JustBas$\}
  \end{center}
  Usaremos $Just$ para denotar el conjunto formado por todas las justificaciones.

  Cabe destacar que los elementos de $Just$ son palabras del alfabeto formado por los siguientes simbolos:
  \begin{center}
  (\ )\ ,\ 0\ 1\ 2\ 3\ 4\ 5\ 6\ 7\ 8\ 9\ A\ B\ C\ D\ E\ G\ H\ I\ J\ L\ M\ N\ O\ P\ Q\ R\ S\ T\ U\ V\ X\ Z
  \end{center}
\end{definition}
\subsection{Concatenaciones balanceadas de justificaciones}
\begin{lemma}
  Sea $\mathbf{J}\in Just^+$. Hay unicos $n\geq1$ y $\succession{J}{1}{n}\in Just$ tales que $\mathbf{J}=\succession[]{J}{1}{n}$
\end{lemma}
\noproof
\begin{definition}
  Dada $\mathbf{J}\in Just^+$, usaremos $n(\mathbf{J})$ y \succession{\mathbf{J}}{1}{n(\mathbf{J})} para denotar los unicos $n$ y \succession{J}{1}{n} cuya existencia 
  garantiza el lema anterior.
\end{definition}
\begin{definition}
  Dados numeros naturales $i\leq j$, usaremos $\bloq{i}{j}$ para denotar al conjunto $\{i,i+1,\dots,j\}$. A los conjuntos de la forma $\bloq{i}{j}$
  los llamaremos \emph{bloques}.
\end{definition}

\begin{definition}
  Dada $\mathbf{J}\in Just^+$ definamos:
  $$
  \mathcal{B}^\mathbf{J}=\{\bloq{i}{j} | \exists k : \mathbf{J}_i = \text{HIPOTESIS}\overline{k} \text{ y } \mathbf{J}_j=\text{TESIS}\overline{k}\alpha \text{ para algun } \alpha \in JustBas\}
  $$
\end{definition}

\begin{definition}
  Diremos que $\mathbf{J}\in Just^+$ es \emph{balanceada} si se dan las siguientes:
  \begin{enumerate}
    \item Por cada $k \in \mathbf{N}$ a lo sumo hay un $i$ tal que $\mathbf{J}_i=$HIPOTESIS$\overline{k}$ y a lo sumo hay un $i$ tal que $\mathbf{J}_i=$TESIS$\overline{k}\alpha$, con $\alpha \in JustBas$.
    \item Si $\mathbf{J}_i$=HIPOTESIS$\overline{k}$, entonces hay un $l > i$ tal que $\mathbf{J}_l$=TESIS$\overline{k}\alpha$, con $\alpha\in JustBas$.
    \item Si $\mathbf{J}_i$=TESIS$\overline{k}\alpha$, con $\alpha\in JustBas$, entonces hay un $l < i$ tal que $\mathbf{J}_l$=HIPOTESIS$\overline{k}$.
    \item Si $B_1,B_2\in\mathcal{B}^\mathbf{J}$, entonces $B_1\cap B_2=\emptyset$ o $B_1\subseteq B_2$ o $B_2\subseteq B_1$
  \end{enumerate}
\end{definition}
\subsection{Pares adecuados}
\begin{lemma}
  Sea $\pvarphi\in S^{\tau+}$. Hay unicos $n\geq1$ y $\succession{\varphi}{1}{n}\in S^\tau$ tales que $\pvarphi=\succession[]{\varphi}{1}{n}$.
\end{lemma}
\begin{proof}
  Supongamos existen $n,m\geq1$ tales que $\pvarphi=\succession[]{\varphi}{1}{n}$ y $\pvarphi=\succession[]{\psi}{1}{m}$,
  para algunos $\succession{\varphi}{1}{n}, \succession{\psi}{1}{m} \in S^\tau$. Si $\varphi_1 \neq \psi_1$,
  una es tramo inicial de la otra, y entonces alguna de ellas no es formula (por mordizqueo de formulas). Luego $\varphi_1 = \psi_1$. Esta logica es analoga para 
  cada $\varphi_i$ y $\psi_i$. Luego $n=m$ y las sentencias involucradas son las mismas. 
\end{proof}

\begin{definition}
  Dada $\pvarphi\in S^{\tau+}$, usaremos $n(\pvarphi)$ y \succession{\pvarphi}{1}{n(\pvarphi)} para denotar los unicos 
  $n $ y $\succession{\varphi}{1}{n}$ cuya existencia garantiza el lema anterior.
\end{definition}

\begin{definition}
  Un \emph{par adecuado de tipo $\tau$} es un par $\padec\in S^{\tau+}\times Just^+$ tal que $n(\pvarphi)=n(\mathbf{J})$ y $\mathbf{J}$ es balanceada.

  Si $\bloq{i}{j} \in \mathcal{B}^\mathbf{J}$, entonces $\pvarphi_i$ sera la \emph{hipotesis} del bloque $\bloq{i}{j}$ en $\padec$ y 
  $\pvarphi_j$ sera la \emph{tesis} del bloque $\bloq{i}{j}$ en $\padec$.

  Diremos que $\pvarphi_i$ esta \emph{bajo la hipotesis} $\pvarphi_l$ en $\padec$ o que $\pvarphi_l$ es una 
  \emph{hipotesis de } $\pvarphi_i$ en $\padec$ cuando haya en $\mathcal{B}^\mathbf{J}$ un bloque de la forma $\bloq{l}{j}$ el cual 
  contenga a $i$.
  
  Sean $i,j\in\bloq{1}{n(\pvarphi)}$. Diremos que $i$ es \emph{anterior} a $j$ en $\padec$ si $i < j$ y ademas para todo $B \in \mathcal{B}^\mathbf{J}$ se tiene 
  que $i\in B \Rightarrow j\in B$.
\end{definition}

\subsection{Dependencia de constantes en pares adecuados}
\begin{definition}
  Dadas $e,d\in\mathcal{C}$, diremos que \emph{e depende directamente de d en } $\padec$ si hay numeros $1\leq l\leq j \leq n(\pvarphi)$ tales que:
  \begin{enumerate}
    \item $l$ es anterior a $j$ en $\padec$
    \item $\mathbf{J}_j=\alpha \text{ELECCION}(\overline{l})$, con $\alpha\in\{\varepsilon\}\cup\{\text{TESIS}\overline{k}:k\in\mathbf{N}\}$ y $(\pvarphi_l,\pvarphi_j)\in \text{Elec}^\tau$ via $e$
    \item $d$ ocurre en $\pvarphi_l$
  \end{enumerate} 

  Dados $e,d\in\mathcal{C}$ diremos que \emph{e depende de d} en $\padec$ si existen $\succession{e}{0}{k+1}\in\mathcal{C}$,
  con $k\geq0$ tales que:
  \begin{enumerate}
    \item $e_0=e$ y $e_{k+1}=d$
    \item $e_i$ depende directamente de $e_{i+1}$ en $\padec$, para $i=0,\dots,k$
  \end{enumerate}
\end{definition}
\subsection{Definicion de prueba}
\begin{definition}
  Sea \forder una teoria de primer orden. Sea $\varphi$ una sentencia de tipo $\tau$. Una \emph{prueba de $\varphi$}
  en \forder sera un par adecuado \padec de algun tipo $\tau_1 = (\mathcal{C}\cup\mathcal{C}_1,\mathcal{F},\mathcal{R},a)$ con $\mathcal{C}_1$ finito y disjunto con $\mathcal{C}$ tal que:
  \begin{enumerate}
    \item Cada $\pvarphi_i$ es una sentencia de tipo $\tau_1$
    \item $\pvarphi_{n(\pvarphi)}=\varphi$
    \item Si $\bloq{i}{j}\in\mathcal{B}^\mathbf{J}$, entonces $\pvarphi_{j+1}=(\pvarphi_i\rightarrow\pvarphi_j)$ y $\mathbf{J}_{j+1}=\alpha$CONCLUSION, \\con $\alpha\in\{\varepsilon\}\cup\{\text{TESIS}\overline{k}:k\in\mathbf{N}\}$
    \item Para cada $i=1,\dots,n(\pvarphi)$ se da una de las siguientes:\begin{enumerate}
      \item $\mathbf{J}_i=\text{HIPOTESIS}\overline{k}$ para algun $k\in\mathbf{N}$
      \item $\mathbf{J}_i=\alpha\text{CONCLUSION}$, con $\alpha\in\{\varepsilon\}\cup\{\text{TESIS}\overline{k}:k\in\mathbf{N}\}$ y hay un $j$ tal que $\bloq{j}{i-1}\in\mathcal{B}^\mathbf{J}$\\ y $\pvarphi_i=(\pvarphi_j\rightarrow\pvarphi_{i-1})$
      \item $\mathbf{J}_i=\alpha\text{AXIOMALOGICO}$, con $\alpha\in\{\varepsilon\}\cup\{\text{TESIS}\overline{k}:k\in\mathbf{N}\}$ y $\pvarphi_i$ es un axioma logico de tipo $\tau_1$
      \item $\mathbf{J}_i=\alpha\text{AXIOMAPROPIO}$, con $\alpha\in\{\varepsilon\}\cup\{\text{TESIS}\overline{k}:k\in\mathbf{N}\}$ y $\pvarphi_i\in\Sigma$
      \item $\mathbf{J}_i=\alpha\text{PARTICULARIZACION}(\overline{l})$, con $\alpha\in\{\varepsilon\}\cup\{\text{TESIS}\overline{k}:k\in\mathbf{N}\}$,\\ $l$ anterior a $i$ y $(\pvarphi_l,\pvarphi_i)\in Partic^{\tau_1}$
      \item $\mathbf{J}_i=\alpha\text{COMMUTATIVIDAD}(\overline{l})$, con $\alpha\in\{\varepsilon\}\cup\{\text{TESIS}\overline{k}:k\in\mathbf{N}\}$, \\$l$ anterior a $i$ y $(\pvarphi_l,\pvarphi_i)\in Commut^{\tau_1}$
      \item $\mathbf{J}_i=\alpha\text{ABSURDO}(\overline{l})$, con $\alpha\in\{\varepsilon\}\cup\{\text{TESIS}\overline{k}:k\in\mathbf{N}\}$, \\$l$ anterior a $i$ y $(\pvarphi_l,\pvarphi_i)\in Absur^{\tau_1}$
      \item $\mathbf{J}_i=\alpha\text{EVOCACION}(\overline{l})$, con $\alpha\in\{\varepsilon\}\cup\{\text{TESIS}\overline{k}:k\in\mathbf{N}\}$, \\$l$ anterior a $i$ y $(\pvarphi_l,\pvarphi_i)\in Evoc^{\tau_1}$
      \item $\mathbf{J}_i=\alpha\text{EXISTENCIA}(\overline{l})$, con $\alpha\in\{\varepsilon\}\cup\{\text{TESIS}\overline{k}:k\in\mathbf{N}\}$, \\$l$ anterior a $i$ y $(\pvarphi_l,\pvarphi_i)\in Exist^{\tau_1}$
      \item $\mathbf{J}_i=\alpha\text{CONJUNCIONELIMINACION}(\overline{l})$, con $\alpha\in\{\varepsilon\}\cup\{\text{TESIS}\overline{k}:k\in\mathbf{N}\}$, \\$l$ anterior a $i$ y $(\pvarphi_l,\pvarphi_i)\in ConjElim^{\tau_1}$
      \item $\mathbf{J}_i=\alpha\text{DISJUNCIONINTRODUCCION}(\overline{l})$, con $\alpha\in\{\varepsilon\}\cup\{\text{TESIS}\overline{k}:k\in\mathbf{N}\}$, \\$l$ anterior a $i$ y $(\pvarphi_l,\pvarphi_i)\in DisjElim^{\tau_1}$
      \item $\mathbf{J}_i=\alpha\text{EQUIVALENCIAELIMINACION}(\overline{l})$, con $\alpha\in\{\varepsilon\}\cup\{\text{TESIS}\overline{k}:k\in\mathbf{N}\}$, \\$l$ anterior a $i$ y $(\pvarphi_l,\pvarphi_i)\in EquivElim^{\tau_1}$
      \item $\mathbf{J}_i=\alpha\text{MODUSPONENS}(\overline{l_1},\overline{l_2})$, con $\alpha\in\{\varepsilon\}\cup\{\text{TESIS}\overline{k}:k\in\mathbf{N}\}$, \\$l_1$ y $l_2$ anteriores a $i$ y $(\pvarphi_{l_1},\pvarphi_{l_2},\pvarphi_i)\in ModPon^{\tau_1}$
      \item $\mathbf{J}_i=\alpha\text{CONJUNCIONINTRODUCCION}(\overline{l_1},\overline{l_2})$, con $\alpha\in\{\varepsilon\}\cup\{\text{TESIS}\overline{k}:k\in\mathbf{N}\}$, \\$l_1$ y $l_2$ anteriores a $i$ y $(\pvarphi_{l_1},\pvarphi_{l_2},\pvarphi_i)\in ConjInt^{\tau_1}$
      \item $\mathbf{J}_i=\alpha\text{EQUIVALENCIAINTRODUCCION}(\overline{l_1},\overline{l_2})$, con $\alpha\in\{\varepsilon\}\cup\{\text{TESIS}\overline{k}:k\in\mathbf{N}\}$, \\$l_1$ y $l_2$ anteriores a $i$ y $(\pvarphi_{l_1},\pvarphi_{l_2},\pvarphi_i)\in EquivInt^{\tau_1}$    
      \item $\mathbf{J}_i=\alpha\text{DISJUNCIONINTRODUCCION}(\overline{l_1},\overline{l_2})$, con $\alpha\in\{\varepsilon\}\cup\{\text{TESIS}\overline{k}:k\in\mathbf{N}\}$, \\$l_1$ y $l_2$ anteriores a $i$ y $(\pvarphi_{l_1},\pvarphi_{l_2},\pvarphi_i)\in DisjElim^{\tau_1}$    
      \item $\mathbf{J}_i=\alpha\text{REEMPLAZO}(\overline{l_1},\overline{l_2})$, con $\alpha\in\{\varepsilon\}\cup\{\text{TESIS}\overline{k}:k\in\mathbf{N}\}$, \\$l_1$ y $l_2$ anteriores a $i$ y $(\pvarphi_{l_1},\pvarphi_{l_2},\pvarphi_i)\in Reemp^{\tau_1}$    
      \item $\mathbf{J}_i=\alpha\text{TRANSITIVIDAD}(\overline{l_1},\overline{l_2})$, con $\alpha\in\{\varepsilon\}\cup\{\text{TESIS}\overline{k}:k\in\mathbf{N}\}$, \\$l_1$ y $l_2$ anteriores a $i$ y $(\pvarphi_{l_1},\pvarphi_{l_2},\pvarphi_i)\in Trans^{\tau_1}$    
      \item $\mathbf{J}_i=\alpha\text{DIVISIONPORCASOS}(\overline{l_1},\overline{l_2},\overline{l_3})$, con $\alpha\in\{\varepsilon\}\cup\{\text{TESIS}\overline{k}:k\in\mathbf{N}\}$, \\$l_1$,$l_2$ y $l_3$ anteriores a $i$ y $(\pvarphi_{l_1},\pvarphi_{l_2},\pvarphi_{l_3},\pvarphi_i)\in DivPorCas^{\tau_1}$
      \item $\mathbf{J}_i=\alpha\text{ELECCION}(\overline{l})$, con $\alpha\in\{\varepsilon\}\cup\{\text{TESIS}\overline{k}:k\in\mathbf{N}\}$, $l$ anterior a $i$ y $(\pvarphi_{l},\pvarphi_i)\in Elec^{\tau_1}$\\ via un nombre de constante $e$, el cual no pertenece a $\mathcal{C}$ y no ocurre en \succession{\pvarphi}{1}{i-1}
      \item $\mathbf{J}_i=\alpha\text{GENERALIZACION}(\overline{l})$, con $\alpha\in\{\varepsilon\}\cup\{\text{TESIS}\overline{k}:k\in\mathbf{N}\}$, $l$ anterior a $i$ y $(\pvarphi_{l},\pvarphi_i)\in Generaliz^{\tau_1}$\\ via un nombre de constante $e$, el cual cumple:\begin{enumerate}
        \item $c\not\in\mathcal{C}$
        \item Para cada $u \in \bloq{1}{n(\pvarphi)}$, si $\mathbf{J}_u=\alpha\text{ELECCION}(\overline{v})$, con $\alpha\in\{\varepsilon\}\cup\{\text{TESIS}\overline{k}:k\in\mathbf{N}\}$, entonces no se da que $(\pvarphi_v,\pvarphi_u)\in Elec^{\tau_1}$ via $c$.
        \item $c$ no ocurre en ninguna hipotesis de $\pvarphi_l$
        \item Ningun nombre de constante que ocurra en $\pvarphi_l$ o en sus hipotesis, depende de $c$.
      \end{enumerate}
    \end{enumerate}
  \end{enumerate}
\end{definition}

\begin{remark}
  Dar una prueba formal de $\forall x\forall y \forall z \forall w\ ((x\leq z)\land(y\leq w)\rightarrow \supbin{x}{y} \leq \supbin{z}{w})$ en $Ret$
  \begin{pformal}
    &1.&\quad& (a\leq b)\land(c\leq d)& & & & \text{HIPOTESIS}1\\
    &2.&\quad& a\leq b& & & & \text{CONJUNCIONELIMINACION}(1)\\
    &3.&\quad& c\leq d& & & & \text{CONJUNCIONELIMINACION}(1)\\
    &4.&\quad& \forall x \forall y\ (x\leq \supbin{x}{y} \land y\leq\supbin{x}{y}) & & & & \text{AXIOMAPROPIO}\\
    &5.&\quad& b\leq\supbin{b}{d}& & & & \text{PARTICULARIZACION}^2(4)\\
    &6.&\quad& d\leq\supbin{b}{d}& & & & \text{PARTICULARIZACION}^2(4)\\
    &7.&\quad& \forall x\forall y\forall z\ ((x\leq y\land y\leq z)\rightarrow x\leq z) & & & & \text{AXIOMAPROPIO}\\
    &8.&\quad& ((a\leq b\land b\leq \supbin{b}{d})\rightarrow a\leq \supbin{b}{d}) & & & & \text{PARTICULARIZACION}^3(7)\\
    &9.&\quad& ((c\leq d\land d\leq \supbin{b}{d})\rightarrow c\leq \supbin{b}{d}) & & & & \text{PARTICULARIZACION}^3(7)\\
    &10.&\quad& (a\leq b \land b\leq \supbin{b}{d}) & & & & \text{CONJUNCIONINTRODUCCION}(2,5)\\
    &11.&\quad& (c\leq d \land d\leq \supbin{b}{d}) & & & & \text{CONJUNCIONINTRODUCCION}(3,6)\\
    &12.&\quad& a\leq\supbin{b}{d} & & & & \text{MODUSPONENS}(10, 8)\\    
    &13.&\quad& c\leq\supbin{b}{d} & & & & \text{MODUSPONENS}(11, 9)\\
    &14.&\quad& \forall x\forall y\forall z\ ((x\leq z \land y\leq z)\rightarrow \supbin{x}{y}\leq z) & & & & \text{AXIOMAPROPIO}\\
    &15.&\quad& ((a\leq \supbin{b}{d} \land c\leq \supbin{b}{d})\rightarrow \supbin{a}{c}\leq \supbin{b}{d}) & & & & \text{PARTICULARIZACION}^3(14)\\
    &16.&\quad& (a\leq\supbin{b}{d}  \land c\leq \supbin{b}{d}) & & & & \text{CONJUNCIONINTRODUCCION}(12,13)\\    
    &17.&\quad& \supbin{a}{c}\leq\supbin{b}{d} & & & & \text{TESIS1MODUSPONENS}(16, 15)\\
    &18.&\quad& ((a\leq b)\land(c\leq d)\rightarrow\supbin{a}{c}\leq\supbin{b}{d}) & & & & \text{CONCLUSION}\\
    &19.&\quad&  \forall x\forall y \forall z \forall w\ ((x\leq z)\land(y\leq w)\rightarrow \supbin{x}{y} \leq \supbin{z}{w}) & & & & \text{GENERALIZACION}^4(18)
  \end{pformal}
\end{remark}

\begin{remark}
  Dar una prueba formal de $\forall x \forall y \forall z\ \supbin{(\supbin{x}{y})}{z} \equiv \supbin{x}{(\supbin{y}{z})}$
  \begin{pformal}
    &1.&\quad& \forall x\forall y\ (x\leq \supbin{x}{y}\land y\leq \supbin{x}{y})&& \text{AXIOMAPROPIO}\\
    &2.&\quad& \supbin{a}{b}\leq \supbin{(\supbin{a}{b})}{c} \land c\leq \supbin{(\supbin{a}{b})}{c} && \text{PARTICULARIZACION}^2(1)\\
    &3.&\quad& \forall x\forall y\forall z((x\leq z\land y\leq z)\rightarrow \supbin{x}{y}\leq z)&& \text{AXIOMAPROPIO}\\
    &4.&\quad& ((a\leq \supbin{(\supbin{a}{b})}{c} \land \supbin{b}{c} \leq \supbin{(\supbin{a}{b})}{c}) \rightarrow \supbin{a}{(\supbin{b}{c})} \leq \supbin{(\supbin{a}{b})}{c})&& \text{PARTICULARIZACION}^3(3)\\
    &5.&\quad& (a\leq \supbin{a}{b}\land b\leq \supbin{a}{b})&& \text{PARTICULARIZACION}^2(1)\\
    &6.&\quad& a\leq \supbin{a}{b}&& \text{CONJUNCIONELIMINACION}(5)\\
    &7.&\quad& (\supbin{a}{b}\leq \supbin{(\supbin{a}{b})}{c}\land c \leq \supbin{(\supbin{a}{b})}{c})&& \text{PARTICULARIZACION}^2(1)\\    
    &8.&\quad& \supbin{a}{b}\leq \supbin{(\supbin{a}{b})}{c}&& \text{CONJUNCIONELIMINACION}(7)\\
    &9.&\quad& \forall x\forall y\forall z\ ((x\leq y\land y\leq z)\rightarrow x\leq z)&& \text{AXIOMAPROPIO}\\    
    &10.&\quad& ((a\leq \supbin{a}{b}\land \supbin{a}{b}\leq \supbin{(\supbin{a}{b})}{c})\rightarrow a\leq \supbin{(\supbin{a}{b})}{c})&& \text{PARTICULARIZACION}^3(9)\\
    &11.&\quad& (a\leq \supbin{a}{b} \land \supbin{a}{b}\leq \supbin{(\supbin{a}{b})}{c})&& \text{CONJUNCIONINTRODUCCION}(6,8)\\    
    &12.&\quad& a\leq \supbin{(\supbin{a}{b})}{c}&& \text{MODUSPONENS}(11, 10)\\ 
    &13.&\quad& c\leq \supbin{(\supbin{a}{b})}{c}&& \text{CONJUNCIONELIMINACION}(2)\\ 
    &14.&\quad& (a \leq \supbin{a}{b} \land b\leq \supbin{a}{b})&& \text{PARTICULARIZACION}^2(1)\\ 
    &15.&\quad& b\leq \supbin{a}{b}&& \text{CONJUNCIONELIMINACION}(14)\\ 
    &16.&\quad& ((b\leq \supbin{a}{b}\land \supbin{a}{b}\leq\supbin{(\supbin{a}{b})}{c})\rightarrow b\leq\supbin{(\supbin{a}{b})}{c}) && \text{PARTICULARIZACION}^3(9)\\
    &17.&\quad& (b\leq \supbin{a}{b} \land \supbin{a}{b}\leq \supbin{(\supbin{a}{b})}{c}) && \text{CONJUNCIONINTRODUCCION}(15,8)\\  
    &18.&\quad& b\leq \supbin{(\supbin{a}{b})}{c} && \text{MODUSPONENS}(17,16)\\
    &19.&\quad& ((b\leq \supbin{(\supbin{a}{b})}{c} \land c\leq \supbin{(\supbin{a}{b})}{c})\rightarrow \supbin{b}{c}\leq \supbin{(\supbin{a}{b})}{c}) && \text{PARTICULARIZACION}^3(3)\\
    &20.&\quad& (b\leq \supbin{(\supbin{a}{b})}{c} \land c\leq \supbin{(\supbin{a}{b})}{c}) && \text{CONJUNCIONINTRODUCCION}(18,13)\\
    &21.&\quad&  \supbin{b}{c}\leq \supbin{(\supbin{a}{b})}{c}&& \text{MODUSPONENS}(20,19)\\
    &22.&\quad&  (a\leq \supbin{(\supbin{a}{b})}{c}\land\supbin{b}{c}\leq \supbin{(\supbin{a}{b})}{c})&& \text{CONJUNCIONINTRODUCCION}(12,21)\\
    &23.&\quad& \supbin{a}{(\supbin{b}{c})} \leq \supbin{(\supbin{a}{b})}{c} && \text{MODUSPONENS}(22,4)\\
    &24.&\quad& ((\supbin{a}{b})\leq \supbin{a}{(\supbin{b}{c})} \land c \leq \supbin{a}{(\supbin{b}{c})}) \rightarrow \supbin{(\supbin{a}{b})}{c} \leq \supbin{a}{(\supbin{b}{c})} && \text{PARTICULARIZACION}^3(3)\\
    &25.&\quad& b \leq \supbin{b}{c} \land c \leq \supbin{b}{c} && \text{PARTICULARIZACION}^2(1)\\
    &26.&\quad& c \leq (\supbin{b}{c})&& \text{CONJUNCIONELIMINACION}(25)\\
    &27.&\quad& a \leq \supbin{a}{(\supbin{b}{c})} \land (\supbin{b}{c}) \leq \supbin{a}{(\supbin{b}{c})} && \text{PARTICULARIZACION}^2(1)\\
    &28.&\quad& (\supbin{b}{c}) \leq \supbin{a}{(\supbin{b}{c})}&& \text{CONJUNCIONELIMINACION}(27)\\
    &29.&\quad& c \leq (\supbin{b}{c}) \land (\supbin{b}{c}) \leq \supbin{a}{(\supbin{b}{c})}\rightarrow c\leq \supbin{a}{(\supbin{b}{c})} && \text{PARTICULARIZACION}^3(9)\\
    &30.&\quad&  c \leq (\supbin{b}{c}) \land (\supbin{b}{c}) \leq \supbin{a}{(\supbin{b}{c})} && \text{CONJUNCIONINTRODUCCION}(26,28)\\
    &31.&\quad& c\leq \supbin{a}{(\supbin{b}{c})} && \text{MODUSPONENS}(30,29)\\
    &32.&\quad& a \leq \supbin{a}{(\supbin{b}{c})} \land b \leq \supbin{a}{(\supbin{b}{c})} \rightarrow (\supbin{a}{b}) \leq \supbin{a}{(\supbin{b}{c})} && \text{PARTICULARIZACION}^3(3)\\
    &33.&\quad& a \leq \supbin{a}{(\supbin{b}{c})} \land (\supbin{b}{c}) \leq \supbin{a}{(\supbin{b}{c})} && \text{PARTICULARIZACION}^2(1)\\
    &34.&\quad& a \leq \supbin{a}{(\supbin{b}{c})} && \text{CONJUNCIONELIMINACION}(33)\\
    &35.&\quad& b \leq \supbin{b}{c} \land c \leq \supbin{b}{c} && \text{PARTICULARIZACION}^2(1)\\
    &36.&\quad& b \leq (\supbin{b}{c}) && \text{CONJUNCIONELIMINACION}(33)\\
    &37.&\quad& b \leq (\supbin{b}{c}) \land (\supbin{b}{c}) \leq \supbin{a}{(\supbin{b}{c})}\rightarrow b\leq \supbin{a}{(\supbin{b}{c})} && \text{PARTICULARIZACION}^3(9)\\
    &38.&\quad& b \leq (\supbin{b}{c}) \land (\supbin{b}{c}) \leq \supbin{a}{(\supbin{b}{c})} && \text{CONJUNCIONINTRODUCCION}(36,28)\\
    &39.&\quad& b\leq \supbin{a}{(\supbin{b}{c})} && \text{MODUSPONENS}(38,37)\\
    &40.&\quad& a \leq \supbin{a}{(\supbin{b}{c})} \land b\leq \supbin{a}{(\supbin{b}{c})} && \text{CONJUNCIONINTRODUCCION}(34,39)\\  
    &41.&\quad& (\supbin{a}{b}) \leq \supbin{a}{(\supbin{b}{c})} && \text{MODUSPONENS}(40,32)\\
    &42.&\quad& (\supbin{a}{b})\leq \supbin{a}{(\supbin{b}{c})} \land c \leq \supbin{a}{(\supbin{b}{c})} && \text{CONJUNCIONINTRODUCCION}(41,31)\\
    &43.&\quad& \supbin{(\supbin{a}{b})}{c} \leq \supbin{a}{(\supbin{b}{c})} && \text{MODUSPONENS}(42,24)\\
    &44.&\quad& \forall x\forall y\ ((x\leq y \land y \leq x) \rightarrow x \equiv y) && \text{AXIOMAPROPIO}\\
    &45.&\quad& (\supbin{a}{(\supbin{b}{c})} \leq \supbin{(\supbin{a}{b})}{c}) \land (\supbin{(\supbin{a}{b})}{c} \leq \supbin{a}{(\supbin{b}{c})}) \rightarrow ... && \text{PARTICULARIZACION}^2(44)\\
    &46.&\quad& \supbin{a}{(\supbin{b}{c})} \leq \supbin{(\supbin{a}{b})}{c} \equiv \supbin{(\supbin{a}{b})}{c} \leq \supbin{a}{(\supbin{b}{c})} && \text{MODUSPONENS(43, 45)}\\
  \end{pformal}
\end{remark}

\begin{remark}
  Sea $\tau = (\emptyset, \emptyset, \{rel^2\}, a)$. De una prueba formal que atestigue que:
  $$
  (\{\forall x\exists y\ (rel(x, y)\lor rel(y,x))\},\tau)\proves \forall x(\forall y\neg rel(x,y)\rightarrow\exists z\ rel(z, x))
  $$
  \begin{pformal}
    &1.&\quad& \forall x\exists y\ (rel(x, y)\lor rel(y,x))& & & & \text{AXIOMAPROPIO}\\
    &2.&\quad& \forall y\neg rel(a,y)& & & & \text{HIPOTESIS}1\\
    &3.&\quad& \exists y\ (rel(a, y)\lor rel(y,a))& & & & \text{PARTICULARIZACION}(1)\\
    &4.&\quad& rel(a,b)\lor rel(b,a)& & & & \text{ELECCION}(3)\\
    &5.&\quad& \neg rel(a, b) & & & & \text{PARTICULARIZACION}(2)\\
    &6.&\quad& rel(b, a) & & & & \text{DISJUNCIONELIMINACION}(5, 4)\\
    &7.&\quad& \exists z\ rel(z, a) & & & & \text{TESIS1EXISTENCIA}(6)\\
    &8.&\quad& (\forall y\neg rel(a,y)\rightarrow\exists z\ rel(z, a)) & & & & \text{CONCLUSION}\\
    &9.&\quad& \forall x(\forall y\neg rel(a,y)\rightarrow\exists z\ rel(z, a)) & & & & \text{GENERALIZACION}(8)\\
  \end{pformal}
\end{remark}

\section{El concepto de teorema}
\begin{definition}
  Cuando haya una prueba de $\varphi$ en \forder, diremos que $\varphi$ es un \emph{teorema} de la teoria \forder, y escribiremos $\forder\proves\varphi$.
\end{definition}

\section{Conteo de modelos modulo isomorfismo}
\begin{definition}
  Sea $T$ una teoria de primer orden. Diremos que $T$ \emph{tiene, modulo isomorfismo, exactamente una 
  cantidad n de modelos de m elementos si hay \succession{\mathbf{A}}{1}{n}} estructuras de tipo $\tau$ tales que:
  \begin{enumerate}
    \item Cada $\mathbf{A}_i$ es un modelo de $T$
    \item $|A_i| = m$, para cada $i=1,\dots,n$
    \item $\mathbf{A}_i$ no es isomorfo a $\mathbf{A}_j$, cada vez que $i\neq j$
    \item Si $\mathbf{A}$ es un modelo de la teoria $T$, y $|A| = m$, entonces $\mathbf{A}$ es isomorfo a $\mathbf{A}_i$ para algun i
  \end{enumerate} 
\end{definition}

\begin{remark}
  Sea $\tau$ un tipo cualquiera y sean $\varphi_1, \varphi_2,\varphi_3$ sentencias de tipo $\tau$.
  Dar una prueba formal de $((\varphi_1\land\varphi_2)\rightarrow\varphi_3)\leftrightarrow(\varphi_1\rightarrow(\varphi_2\rightarrow\varphi_3))$
\end{remark}
\begin{proof}
  \begin{pformal}
    &1.&\quad& ((\varphi_1\land\varphi_2)\rightarrow\varphi_3)& & & & \text{HIPOTESIS1}\\
    &2.&\quad& \varphi_1& & & & \text{HIPOTESIS}2\\
    &3.&\quad& \varphi_2& & & & \text{HIPOTESIS}3\\
    &4.&\quad& (\varphi_1 \land \varphi_2)& & & & \text{CONJUNCIONINTRODUCCION}(2,3)\\
    &5.&\quad& \varphi_3 & & & & \text{TESIS3MODUSPONENS(4,1)}\\
    &6.&\quad& (\varphi_2\rightarrow\varphi_3) & & & & \text{TESIS2CONCLUSION}\\
    &7.&\quad& (\varphi_1\rightarrow(\varphi_2\rightarrow\varphi_3)) & & & & \text{TESIS1CONCLUSION}\\
    &8.&\quad& ((\varphi_1\land\varphi_2)\rightarrow\varphi_3)\rightarrow(\varphi_1\rightarrow(\varphi_2\rightarrow\varphi_3)) & & & & \text{CONCLUSION}\\    
    &9.&\quad& (\varphi_1\rightarrow(\varphi_2\rightarrow\varphi_3)) & & & & \text{HIPOTESIS}4\\    
    &10.&\quad& (\varphi_1\land\varphi_2) & & & & \text{HIPOTESIS}5\\        
    &11.&\quad& \varphi_1 & & & & \text{CONJUNCIONELIMINACION}(10)\\        
    &12.&\quad& \varphi_2 & & & & \text{CONJUNCIONELIMINACION}(10)\\        
    &13.&\quad& (\varphi_2\rightarrow\varphi_3) & & & & \text{MODUSPONENS}(11,9)\\        
    &14.&\quad& \varphi_3 & & & & \text{TESIS5MODUSPONENS}(12,13)\\        
    &15.&\quad& ((\varphi_1\land\varphi_2)\rightarrow\varphi_3) & & & & \text{TESIS4CONCLUSION}\\        
    &16.&\quad& ((\varphi_1\rightarrow(\varphi_2\rightarrow\varphi_3))\rightarrow((\varphi_1\land\varphi_2)\rightarrow\varphi_3)) & & & & \text{CONCLUSION}\\        
    &17.&\quad& ((\varphi_1\land\varphi_2)\rightarrow\varphi_3)\leftrightarrow(\varphi_1\rightarrow(\varphi_2\rightarrow\varphi_3)) & & & & \text{EQUIVALENCIAINTRODUCCION}(8, 16)\\        
  \end{pformal}
\end{proof}
