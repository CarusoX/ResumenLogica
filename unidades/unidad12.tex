% !TeX root = ../resumen.tex

\begin{document}

\section{La aritmetica de Peano}

\begin{definition}
  Sea $\tau_A = \aritp$. Denotaremos con \pomega a la estructura de tipo $\tau_A$
  que tiene a $\omega$ como universo e interpreta los nombres $\tau_A$ en la manera usual, es decir:
  \begin{alignat*}{4}
    &\ 0^{\pomega} &\ =&\ 0\\
    &\ 1^{\pomega} &\ =&\ 1\\
    &\leq^{\pomega} &\ =&\ \{(n,m)\in \omega^2: n\leq m\}\\
    &+^{\pomega}(n,m) &\ =&\ n+m, \text{ para cada } n,m\in\omega\\
    &\ .^{\pomega}(n,m) &\ =&\ n.m, \text{ para cada } n,m\in\omega
  \end{alignat*}

  Sea $\Sigma$ el conjunto formado por las siguientes sentencias:\begin{enumerate}
    \item $\forall x_1 \forall x_2 \forall x_3\ x_1+(x_2+x_3)\equiv(x_1+x_2)+x_3$
    \item $\forall x_1 \forall x_2\ x_1+x_2\equiv x_2+x_1$
    \item $\forall x_1 \forall x_2 \forall x_3\ x_1.(x_2.x_3)\equiv(x_1.x_2).x_3$
    \item $\forall x_1 \forall x_2\ x_1.x_2\equiv x_2.x_1$ 
    \item $\forall x_1\ x_1+0\equiv x_1$
    \item $\forall x_1\ x_1.0\equiv 0$
    \item $\forall x_1\ x_1.1\equiv 1$
    \item $\forall x_1 \forall x_2 \forall x_3\ x_1.(x_2+x_3)\equiv(x_1.x_2)+(x_1.x_3)$
    \item $\forall x_1 \forall x_2 \forall x_3\ (x_1+x_3\equiv x_2+x_3\rightarrow x_1\equiv x_2)$
    \item $\forall x_1\ x_1\leq x_1$
    \item $\forall x_1 \forall x_2 \forall x_3\ ((x_1\leq x_2\land x_2\leq x_3)\rightarrow x_1\leq x_3)$
    \item $\forall x_1 \forall x_2\ ((x_1\leq x_2 \land x_2\leq x_1)\rightarrow x_1\equiv x_2)$
    \item $\forall x_1 \forall x_2\ (x_1\leq x_2 \lor x_2\leq x_1)$
    \item $\forall x_1 \forall x_2\ (x_1\leq x_2 \leftrightarrow \exists x_3\ x_2\equiv x_1+x_3)$
    \item $0<1$
  \end{enumerate}

  Es facil ver que estas sentencias son satisfechas por \pomega, por lo cual \pomega es un modelo de la teoria $(\Sigma, \tau_A)$. Definamos
  $$
  Verd_{\pomega} = \{\varphi: S^{\tau_A}:\pomega\models\varphi\}
  $$

  \end{definition}
  \begin{remark}
    
  Sea $\mathbf{Q}^{\geq 0}$ la estructura de tipo $\tau_A$ que tiene a ${r\in\mathbf{Q}:r\geq0}$ como universo e interpreta
  los nombres de $\tau_A$ de la manera usual. Notese que $\mathbf{Q}^{\geq 0}$ tambien es un modelo de la teoria $(\Sigma, \tau_A)$ definida justo antes.
  Pero entonces los teoremas de $(\Sigma,\tau_A)$ deben ser verdaderos en $\mathbf{Q}^{\geq 0}$, pero la sentencia 
  $\forall x (x\leq1\rightarrow (x\equiv0\land x\equiv1))$ es falsa en $\mathbf{Q}^{\geq 0}$, por lo cual no es 
  un teorema de $(\Sigma,\tau_A)$, sin embargo pertenece a $Verd_{\pomega}$.
  
  Es decir, los axiomas que habiamos definido antes son demasiado generales y deberiamos agregar axiomas mas caracteristicos de la estructura
  particular de \pomega.
  \end{remark}

  \begin{definition}
    Dada una formula $\psi\in F^{\tau_A}$ y variables \succession{v}{1}{n+1}, con $n\geq0$, tales que $Li(\psi)\subseteq\{\succession{v}{1}{n+1}\}$ y $v_i\neq v_j$ siempre que $i\neq j$,
    denotaremos con $Ind_{\psi,\succession{v}{1}{n+1}}$ a la siguiente sentencia de tipo $\tau_A$
    $$
    \succession{\forall v}{1}{n}\ ((\psi(\vec{v},0)\land\forall v_{n+1}\ (\psi(\vec{v},v_{n+1})\rightarrow\psi(\vec{v},+(v_{n+1},1))))\rightarrow\forall v_{n+1}\ \psi(\vec{v},v_{n+1}))
    $$
    donde suponemos que hemos declarado $\psi=_d\psi(\succession{v}{1}{n+1})$.

    Sea $\Sigma_A$ el conjunto que resulta de agregarla al $\Sigma$ definido anteriormente todas las sentencias de la forma $Ind_{\psi,\succession{v}{1}{n+1}}$.
    La teoria $(\Sigma_A,\tau_A)$ sera llamada \emph{Aritmetica DE Peano} y la denotaremos con $Arit$.
  \end{definition}

  \begin{lemma}
    \pomega es un modelo de $Arit$
  \end{lemma}

  \begin{remark}
    $Ind_{\psi,\succession{v}{1}{n+1}}$ es verdadera en \pomega.
    \begin{proof}
       Supongamos que no. Entonces existen valores $\vec{v}=(\succession{v}{1}{n})$ tal que 
       $$
       \pomega\not\models ((\psi(\vec{v},0)\land(\forall v_{n+1}\ (\psi(\vec{v}, v_{n+1})\rightarrow\psi(\vec{v},+(v_{n+1},1))))\rightarrow\forall v_{n+1}\ \psi(\vec{v},v_{n+1}))[(\succession{v}{1}{n},\dots)]
       $$
       Pero entonces $\pomega\models(\psi(\vec{v},0)\land(\forall v_{n+1}\ (\psi(\vec{v}, v_{n+1})\rightarrow\psi(\vec{v},+(v_{n+1},1))))$ y 
       $\pomega\not\models\forall v_{n+1}\ \psi(\vec{v},v_{n+1})$, por lo tanto existe un valor para $v_{n+1}$ que no satisface $\psi(\vec{v},v_{n+1})$.
       Es facil ver que ese valor no puede ser 0, y por lo tanto no puede ser 1, y por lo tanto no puede ser 2, $\dots$. \abs
      
      \end{proof}
  \end{remark}

  \begin{remark}
    $\mathbf{Q}^{\geq 0}$ no es un modelo de $Arit$
    \begin{proof}
      TODO
    \end{proof}
  \end{remark}

  \begin{definition}
    Definimos la funcion \functype{\widehat{\ \ }}{\omega}{\{(\ )\ ,\ +\ 0\ 1\}^*} de la siguiente manera:
    \begin{alignat*}{3}
      &\widehat{0} &\ =&\ 0\\
      &\widehat{1} &\ =&\ 1\\
      &\widehat{n+1} &\ =&\ +(\widehat{n},1),\text{ para cada }n\geq1
    \end{alignat*}
  \end{definition}

  \begin{proposition}
    Hay un modelo de Arit el cual no es isomorfo a \pomega
  \end{proposition}

  \begin{lemma}
    Las siguientes sentencias son teoremas de la aritmetica de Peano:
    \begin{enumerate}
      \item $\forall x\ 0\leq x$
      \item $\forall x\ (x\leq0\rightarrow x\equiv 0)$
      \item $\forall x\forall y\ (x+y\equiv0\rightarrow x\equiv0\land y\equiv0)$
      \item $\forall x\ (\neg(x\equiv0)\rightarrow\exists z(x\equiv z+1))$
      \item $\forall x\forall y\ (x<y\rightarrow x+1\leq y)$
      \item $\forall x\forall y\ (x<y+1\rightarrow x\leq y)$
      \item $\forall x\forall y\ (x\leq y+1\rightarrow (x\leq y\lor x\equiv y+1))$
    \end{enumerate}
  \end{lemma}
  \begin{proof}
    \begin{singlespace}
      Prueba de (1) - TODO
    \end{singlespace}
    \begin{singlespace}
      Prueba de (2)
      \begin{flalign*}
      &1.&\quad& x_0\leq0 & & & & \text{HIPOTESIS}1\\
      &2.&\quad& \forall x\ 0\leq x & & & & \text{TEOREMA}\\
      &3.&\quad& 0\leq x_0 & & & & \text{PARTICULARIZACION}(2)\\
      &4.&\quad& x_0\leq0\land0\leq x_0 & & & & \text{CONJUNCIONINTRODUCCION}(1,3)\\
      &5.&\quad& \forall x_1\forall x_2\ ((x_1\leq x_2\land x_2\leq x_1)\rightarrow x_1\equiv x_2) & & & & \text{AXIOMAPROPIO}\\
      &6.&\quad& \forall x_2 ((x_0\leq x_2\land x_2\leq x_0)\rightarrow x_0\equiv x_2) & & & & \text{PARTICULARIZACION}(5)\\
      &7.&\quad& ((x_0\leq 0\land 0\leq x_0)\rightarrow x_0\equiv 0) & & & & \text{PARTICULARIZACION}(6)\\
      &8.&\quad& x_0\equiv0 & & & & \text{TESIS1MODUSPONENS}(4,7)\\
      &9.&\quad& (x_0\leq0\rightarrow x_0\equiv 0) & & & & \text{CONCLUSION}\\
      &10.&\quad& \forall x\ (x\leq 0\rightarrow x\equiv 0) & & & & \text{GENERALIZACION}(9)\\
    \end{flalign*}
  \end{singlespace}
  \begin{singlespace}
  Prueba de (3) \begin{flalign*}
      &1.&\quad& x_0+y_0\equiv 0 & & & & \text{HIPOTESIS}1\\
      &2.&\quad& 0\equiv x_0+y_0 & & & & \text{COMMUTATIVIDAD}(1)\\
      &3.&\quad& \exists x_3\ (0\equiv x_0+x_3) & & & & \text{EXISTENCIA}(2)\\
      &4.&\quad& \forall x_1\forall x_2\ (x_1\leq x_2\leftrightarrow \exists x_3\ x_2\equiv x_1+x_3) & & & & \text{AXIOMAPROPIO}\\
      &5.&\quad& (x_0\leq0\leftrightarrow\exists x_3\ 0\equiv x_0+x_3) & & & & \text{PARTICULARIZACION}^2(5)\\
      &6.&\quad& x_0\leq0 & & & & \text{REEMPLAZO}(5,3)\\
      &7.&\quad& \forall x\ (x\leq0\rightarrow x\equiv 0) & & & & \text{TEOREMA}\\
      &8.&\quad& (x_0\leq0\rightarrow x_0\equiv0) & & & & \text{PARTICULARIZACION}(7)\\
      &9.&\quad& x_0\equiv0 & & & & \text{MODUSPONENS}(6,8)\\
      &10.&\quad& 0 + y_0 \equiv 0 & & & & \text{REEMPLAZO}(9,1)\\
      &11.&\quad& \forall x_1\ x_1+0\equiv x_1 & & & & \text{AXIOMAPROPIO}\\
      &12.&\quad& y_0+0\equiv y_0 & & & & \text{PARTICULARIZACION}(11)\\
      &13.&\quad& \forall x_1\forall x_2\ x_1+x_2\equiv x_2+x_1 & & & &\text{AXIOMAPROPIO}\\
      &14.&\quad& y_0+0\equiv 0+y_0 & & & &\text{PARTICULARIZACION}^2(13)\\
      &15.&\quad& 0+y_0\equiv y_0 & & & & \text{REEMPLAZO}(14, 12)\\
      &16.&\quad& y_0\equiv0 & & & & \text{TRANSITIVIDAD}(15, 10)\\
      &17.&\quad& x_0\equiv0\land y_0\equiv 0 & & & & \text{TESIS1CONJUNCIONINTRODUCCION}(12,16)\\
      &18.&\quad& (x_0+y_0\equiv0\rightarrow(x_0\equiv0\land y_0\equiv0)) & & & & \text{CONCLUSION}\\
      &19.&\quad& \forall x\forall y\ (x+y\equiv0 \rightarrow(x\equiv0\land y\equiv0)) & & & & \text{GENERALIZACION}^2(18)\\
    \end{flalign*}
    \begin{singlespace}
      Prueba de (4) - TODO
    \end{singlespace}
  \end{singlespace}
  \begin{singlespace}
    Prueba de (5) - TODO
  \end{singlespace}
  \begin{singlespace}
    Prueba de (6) - TODO
  \end{singlespace}
  \begin{singlespace}
    Prueba de (7)
    \begin{flalign*}
      &1.&\quad& x_0\leq y_0+1 & & & & \text{HIPOTESIS}1\\
      &2.&\quad& x_0\leq y_0 & & & & \text{HIPOTESIS}2\\
      &3.&\quad& (x_0<y_0+1\rightarrow x_0+1\leq y_0+1) & & & & \text{PARTICULARIZACION}^2(2)\\
      &4.&\quad& x_0+1\leq y_0+1 & & & & \text{TESIS1MODUSPONENS}(1,3)\\
      &5.&\quad& (x_0<y_0+1\rightarrow x_0+1\leq y_0+1)& & & & \text{CONCLUSION}\\
      &6.&\quad& x_0\leq0 & & & & \text{REEMPLAZO}(5,3)\\
      &7.&\quad& \forall x\ (x\leq0\rightarrow x\equiv 0) & & & & \text{TEOREMA}\\
      &8.&\quad& (x_0\leq0\rightarrow x_0\equiv0) & & & & \text{PARTICULARIZACION}(7)\\
      &9.&\quad& x_0\equiv0 & & & & \text{MODUSPONENS}(6,8)\\
      &10.&\quad& 0 + y_0 \equiv 0 & & & & \text{REEMPLAZO}(9,1)\\
      &11.&\quad& \forall x_1\ x_1+0\equiv x_1 & & & & \text{AXIOMAPROPIO}\\
      &12.&\quad& y_0+0\equiv y_0 & & & & \text{PARTICULARIZACION}(11)\\
      &13.&\quad& \forall x_1\forall x_2\ x_1+x_2\equiv x_2+x_1 & & & &\text{AXIOMAPROPIO}\\
      &14.&\quad& y_0+0\equiv 0+y_0 & & & &\text{PARTICULARIZACION}^2(13)\\
      &15.&\quad& 0+y_0\equiv y_0 & & & & \text{REEMPLAZO}(14, 12)\\
      &16.&\quad& y_0\equiv0 & & & & \text{TRANSITIVIDAD}(15, 10)\\
      &17.&\quad& x_0\equiv0\land y_0\equiv 0 & & & & \text{TESIS1CONJUNCIONINTRODUCCION}(12,16)\\
      &18.&\quad& (x_0+y_0\equiv0\rightarrow(x_0\equiv0\land y_0\equiv0)) & & & & \text{CONCLUSION}\\
      &19.&\quad& \forall x\forall y\ (x+y\equiv0 \rightarrow(x\equiv0\land y\equiv0)) & & & & \text{GENERALIZACION}^2(18)\\
    \end{flalign*}
  \end{singlespace}
\end{proof}

\end{document}